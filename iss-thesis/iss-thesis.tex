% -----------------------------------------------------------------
% Vorlage fuer Ausarbeitungen von
% Bachelor- und Masterarbeiten am ISS
% 
% Template for written reports or master theses at the ISS
% 
% For use with compilers pdflatex or latex->dvi2ps->ps2pdf.
%
% -----------------------------------------------------------------
% README, STUDENT USERS:
% We highly appreciate students using this template _AS IS_,period. 
% The document provides adjustable document preferences, 
% student information settings and typography definitions. Look for
% code delimited by *** ***
%
% The short explanation: it's the ISS common standard and 
% 	it's battle tested.
% The long explanation: 
%	We do not want you to go through the document and tweak the 
%	package options, layout parameters and line skips here and 
%	there and waste hours. We are providing this template such 
%	that you can fully concentrate on filling in the much more 
%	important _contents_ of your thesis.
%
% If you have serious needs on extra packages or design 
% modifications, talk to your supervisor _before_ modifying 
% the template.
% Similarly, we're happy if you give your supervisor a hint on any 
% errors in this template.
%
% -----------------------------------------------------------------
% History:
% Jan Scheuing,   04.03.2002
% Markus Buehren, 20.12.2004
% last changes:   10.01.2008 (removed unused packages), 
% 		07.08.2009 (added IEEEtran_LSS.bst file)
% 		02.05.2011 removed matriculation number from cover page
% Martin Kreissig, 25.01.2012: all eps/ps parts removed for 
% 				pdflatex to work properly
% Peter Hermannstaedter, 14.08.2012: fusion of versions for 
% 		latex/dvi/ps/pdf and pdflatex, additional comments,
% 		unification of document flags and student options
% Florian Liebgott, 12.03.2015: bug fixes, removal of obsolete options,
%		switch to UTF-8
% Florian Liebgott, 20.05.2015: fixed encoding problem on title page
% Florian Liebgott, 24.01.2017: changed deprecated font commands (like
%		\sl) to up-to-date commands to be compatible with
%		current TeX distributions.
% Felix Wiewel, 30.08.2021: Replace obsolete scrpage2 with scrlayer-scrpage
%
% -----------------------------------------------------------------
% If you experience any errors caused by this template, please
% contact Florian Liebgott (florian.liebgott@iss.uni-stuttgart.de)
% or your supervisor, so we can fix the errors.
% -----------------------------------------------------------------


\documentclass[12pt,DIV14,BCOR12mm,a4paper,footinclude=false,headinclude,parskip=half-,twoside,openright,cleardoublepage=empty,toc=index,bibliography=totoc,listof=totoc]{scrreprt}
% encoding needs to be defined here, otherwise umlauts on the titelpage won't work.
\usepackage[utf8]{inputenc}
%
%
%
% *****************************************************************
% -------------------> document preferences here <-----------------
% *****************************************************************
% Uncomment the settings you like and comment the settings you don't
% like.

% Language: 
% affects generic titles, Figure term, titlepage and bibliography
% (Note:if you switch the language, compile tex and bib >2 times)
\def \doclang{english} 	% For theses/reports in English
%\def \doclang{german} 		% For theses/reports in German

% Hyperref links in the document:
\def \colortype{color} % links with colored text
%\def \colortype{bw} 	% plain links, standard text color (e.g. for print)
%\def \colortype{boxed} % links with colored boxes
% *****************************************************************
%
%
%
% *****************************************************************
% --------------> put student information here <------------------
% *****************************************************************
% Please fill in all items denoted by "to be defined (TBD)"
\def \deworktitle{Semantisch gesteuerte Videogenerierung f\"ur autonomes Fahren mittels latenter Diffusionsmodelle}        % German title/translation
\def \enworktitle{Semantically Controlled Video Generation for Autonomous Driving using Latent Diffusion Models}        % English title/translation
\def \tutor{Khaled Seyam}
\def \student{Mohammed Jaseel Kunnathodika}
\def \worksubject{Master Thesis}  % type and number (S/Dxxxx) of your thesis
\def \startdate{01.09.2025}
\def \submission{28.02.2026}
\def \signagedate{28.02.2026}   % Date of signature of declaration on last page
\def \keywords{Video Generation, Diffusion Models, Semantic Segmentation, Autonomous Driving, Variational Autoencoder, ControlNet, KITTI-360}
\def \abstract{This thesis presents a two-stage pipeline for controllable video generation of autonomous driving scenes, where semantic segmentation maps serve as the control signal. In the first stage, a diffusion model predicts future semantic video sequences from an initial RGB frame. In the second stage, a ControlNet-conditioned video diffusion model generates photorealistic RGB videos guided by the predicted semantic maps. A central contribution is the design and training of a Semantic-Native Variational Autoencoder (VAE) that encodes discrete semantic label maps into the continuous latent space of a pretrained Stable Video Diffusion model, achieving 89.7\% mean Intersection-over-Union on KITTI-360 validation data. The full pipeline is trained and evaluated on the KITTI-360 dataset, yielding a Fr\'echet Inception Distance of 35.74 and a Fr\'echet Video Distance of 392.10. The results demonstrate that semantic control provides a viable and interpretable mechanism for generating realistic synthetic driving video data.}

% *****************************************************************
%


\usepackage{amsmath}
\usepackage{amsfonts}
\usepackage{ifthen}
\ifthenelse{\equal{\doclang}{german}}{
	\usepackage[ngerman]{babel} %german version
	\def \maintitle{\deworktitle}
	\def \translatedtitle{\enworktitle}
	% set , to decimal and . to thousands separator, if German language is used
	\DeclareMathSymbol{,}{\mathord}{letters}{"3B}
	\DeclareMathSymbol{.}{\mathpunct}{letters}{"3A}
	}{
	%english version
	\def \maintitle{\enworktitle}
	\def \translatedtitle{\deworktitle}
	}
\usepackage{txfonts} % Times-Fonts
\usepackage[T1]{fontenc}
\usepackage{color}
\usepackage[headsepline]{scrlayer-scrpage} % Headings

\usepackage{graphicx}
\usepackage[format=hang]{caption}       % for hanging captions
\usepackage{subfig}                     % for subfigures
\usepackage{wrapfig}                    % for figures floating in text, alternatively you can use >>floatflt<<
\usepackage{booktabs}                   % nice looking tables (for tables with ONLY horizontal lines)

%%%%% Tikz / PGF - drawing beautiful graphics and plots in Latex
% \usepackage{tikz}
% \usetikzlibrary{plotmarks}              % larger choice of plot marks
% \usetikzlibrary{arrows}                 % larger choice of arrow heads
% % ... insert other libraries you need
% \usepackage{pgfplots}
% % set , to decimal and . to thousands separator for plots, if German language is used
% \ifthenelse{\equal{\doclang}{german}}{
% \pgfkeys{/pgf/number format/set decimal separator={,}}
% \pgfkeys{/pgf/number format/set thousands separator={.}}
% }{}
%%%%%%

\ifthenelse{\equal{\colortype}{color}}{
	% colored text version:
	\usepackage[colorlinks,linkcolor=blue]{hyperref}
	\newcommand{\bugfix}{\color{white}{\texttt{\symbol{'004}}}} % Bug-Fix Umlaute in Verbatim
}{
	\ifthenelse{\equal{\colortype}{boxed}}{
		% colored box version:
		\usepackage{hyperref}
		\newcommand{\bugfix}{\color{white}{\texttt{\symbol{'004}}}} % Bug-Fix Umlaute in Verbatim
	}{
		% monochrome version:
		\usepackage[hidelinks]{hyperref}
		\newcommand{\bugfix}{\color{white}{\texttt{\symbol{'004}}}} % Bug-Fix Umlaute in Verbatim
	}
}

% Layout and Headings
\pagestyle{scrheadings}
\automark{chapter}
\clearscrheadfoot
\lehead[]{\pagemark~~\headmark}
\rohead[]{\headmark~~\pagemark}
\renewcommand{\chaptermark}[1]{\markboth {\normalfont\slshape \hspace{8mm}#1}{}}
\renewcommand{\sectionmark}[1]{\markright{\normalfont\slshape \thesection~#1\hspace{8mm}}}
\addtolength{\textheight}{15mm}
\parindent0ex
\setlength{\parskip}{5pt plus 2pt minus 1pt}
\renewcommand*{\pnumfont}{\normalfont\slshape} % Seitenzahl geneigt
\renewcommand*{\sectfont}{\bfseries} % Kapitelueberschrift nicht Helvetica

% Settings for PDF document
\pdfstringdef \studentPDF {\student} 
\pdfstringdef \worktitlePDF {\maintitle}
\pdfstringdef \worksubjectPDF {\worksubject}
\hypersetup{pdfauthor=\studentPDF, 
            pdftitle=\worktitlePDF,
            pdfsubject=\worksubjectPDF}

% Title page
\titlehead{
	\includegraphics[width=20mm]{university-logo}
	\hspace{6mm}
	\ifthenelse{\equal{\doclang}{german}}{
		\begin{minipage}[b]{.6\textwidth}
		{\Large Universit\"at Stuttgart } \\
		Institut f\"ur Signalverarbeitung und Systemtheorie\\
		Professor Dr.-Ing. B. Yang \vspace{0pt}
		\end{minipage}
	}{
		\begin{minipage}[b]{.6\textwidth}
		{\Large University of Stuttgart } \\
		Institute for Signal Processing and System Theory\\
		Professor Dr.-Ing. B. Yang \vspace{0pt}
		\end{minipage}
	}
	\hspace{1mm}
	\includegraphics[width=28mm]{isslogocolor}
}
\subject{\worksubject\vspace*{-5mm}} % Art und Nummer der Arbeit
\title{\maintitle}%\\ \Large{\subtitle}}
\subtitle{\translatedtitle}
\author{
\large
  \ifthenelse{\equal{\doclang}{german}}{
  \begin{tabular}{rp{7cm}}
    \Large 
    Autor:      & \Large \student \vspace*{2mm}\\
    Ausgabe:    & \startdate \\
    Abgabe:     & \submission \vspace*{3mm}\\
    Betreuer:   & \tutor \vspace*{2mm}\\
    Stichworte: & \keywords
  \end{tabular}
  }{
  \begin{tabular}{rp{7cm}}
    \Large 
    Author:             & \Large \student \vspace*{2mm}\\
    Date of work begin: & \startdate \\
    Date of submission: & \submission \vspace*{3mm}\\
    Supervisor:         & \tutor \vspace*{2mm}\\
    Keywords:           & \keywords
  \end{tabular}
  }
  \bugfix
}
\date{}
\publishers{\normalsize
  \begin{minipage}[t]{.9\textwidth}
    \abstract
  \end{minipage}
}

\numberwithin{equation}{chapter} 
\sloppy 

%
%
%
% *****************************************************************
% --------------> put typography definitions here <----------------
% *****************************************************************
% colors
\definecolor{darkblue}{rgb}{0,0,0.4}

% declarations
\newcommand{\matlab}{\textsc{Matlab}\raisebox{1ex}{\tiny{\textregistered}} }
% Integers, natural, real and complex numbers
\newcommand{\Z}{\mathbb{Z}}
\newcommand{\N}{\mathbb{N}}
\newcommand{\R}{\mathbb{R}}
\newcommand{\C}{\mathbb{C}}
% expectation operator
\newcommand{\E}{\operatorname{E}}
% imaginary unit
\newcommand{\im}{\operatorname{j}}
% Euler's number with exponent as parameter, e.g. \e{\im\omega}
\newcommand{\e}[1]{\operatorname{e}^{\,#1}}
% short command for \operatorname{}
\newcommand{\op}[1]{\operatorname{#1}}

% unknown hyphenation rules
\hyphenation{Im-puls-ant-wort Im-puls-ant-wort-ko-ef-fi-zien-ten
Pro-gramm-aus-schnitt Mi-kro-fon-sig-nal}
% *****************************************************************
%
\begin{document}

% title and table of contents
\pagenumbering{alph}
\maketitle
\cleardoublepage
\pagenumbering{roman} % roman numbering for table of contents
\tableofcontents
\cleardoublepage
\setcounter{page}{1}
\pagenumbering{arabic} % arabic numbering for rest of document

% *****************************************************************
% -------------------> start writing here <------------------------

% =================================================================
% CHAPTER 1: INTRODUCTION
% =================================================================
\chapter{Introduction}
\label{ch:introduction}

\section{Motivation}
\label{sec:motivation}

The development and validation of autonomous driving systems requires vast quantities of diverse and accurately annotated video data. Real-world data collection is expensive, time-consuming, and inherently limited in its ability to capture rare but safety-critical scenarios such as pedestrian crossings, adverse weather conditions, and unusual traffic configurations. Synthetic data generation has therefore emerged as a complementary strategy that can provide virtually unlimited, perfectly labeled training data for perception, planning, and prediction modules in self-driving stacks~\cite{li2023drivingdiffusion, hu2023gaia}.

Traditional approaches to synthetic driving data rely on game engines or dedicated simulation platforms. While these methods provide high controllability, they suffer from a persistent \emph{domain gap} between rendered and real imagery. Recent advances in generative modeling---particularly diffusion-based methods~\cite{ho2020denoising, rombach2022highresolution}---have opened a promising alternative: learning the distribution of real driving videos directly from data and then sampling new, photorealistic sequences from the learned model.

A key challenge in this paradigm is \emph{controllability}. Unconditional or text-conditioned video diffusion models can produce visually plausible driving clips, but they offer no mechanism to specify the precise layout of the scene, the trajectories of vehicles, or the spatial arrangement of static elements such as roads, buildings, and vegetation. For autonomous driving applications, the ability to prescribe the semantic layout of every frame is essential---it allows researchers to systematically vary scene composition, test edge cases, and generate ground-truth annotations automatically.

Semantic segmentation maps provide a natural control signal for this purpose. A semantic map assigns each pixel a class label (e.g., road, car, building, sky), thereby offering a dense, interpretable, and spatially precise description of the scene. If a generative model can faithfully translate a sequence of semantic maps into a photorealistic video, then full control over the generated scene is achieved simply by specifying or predicting the semantic maps.

This thesis addresses exactly this problem: \emph{how to generate controllable, high-fidelity driving videos conditioned on semantic segmentation maps, using latent video diffusion models.}

\section{Research Objectives}
\label{sec:objectives}

The primary objective of this work is to design, implement, and evaluate a two-stage pipeline for semantically controlled video generation in the autonomous driving domain. The specific research objectives are:

\begin{enumerate}
    \item \textbf{Semantic VAE Design:} Develop a Variational Autoencoder (VAE) capable of encoding discrete semantic segmentation maps into the continuous latent space of a pretrained video diffusion model, while preserving class boundaries and semantic fidelity.
    
    \item \textbf{Stage~1 --- Semantic Video Prediction:} Train a diffusion model that, given an initial RGB frame, predicts future semantic segmentation video sequences.
    
    \item \textbf{Stage~2 --- Semantic-to-Video Generation:} Train a ControlNet-conditioned video diffusion model that generates photorealistic RGB driving videos guided by the semantic maps produced in Stage~1.
    
    \item \textbf{Evaluation:} Quantitatively assess the pipeline using standard metrics including Fr\'echet Inception Distance (FID), Fr\'echet Video Distance (FVD), and Mean Intersection-over-Union (mIoU), and compare with related approaches.
\end{enumerate}

\section{Contributions}
\label{sec:contributions}

The main contributions of this thesis are:

\begin{itemize}
    \item A \textbf{Semantic-Native VAE} architecture that bypasses the RGB bottleneck of pretrained image VAEs by introducing trainable semantic stem and head modules around a frozen Stable Video Diffusion VAE core. The proposed model achieves 89.7\% mIoU on KITTI-360 validation data with only $\sim$200K trainable parameters.
    
    \item A \textbf{boundary-weighted cross-entropy loss} combined with a \textbf{Dice loss} for training the Semantic VAE, which addresses the critical problem of boundary degradation and class imbalance in semantic reconstruction.
    
    \item A complete \textbf{two-stage diffusion pipeline} for semantically controlled video generation, adapted from the Ctrl-V framework~\cite{luo2025ctrlv}, where bounding-box conditioning is replaced by dense semantic map conditioning.
    
    \item A thorough \textbf{experimental evaluation} on the KITTI-360 dataset~\cite{liao2022kitti360}, including ablation studies on loss functions, architectural choices, and training configurations.
\end{itemize}

\section{Thesis Outline}
\label{sec:outline}

The remainder of this thesis is organized as follows:

\textbf{Chapter~\ref{ch:related_work} -- Related Work} reviews the theoretical foundations and prior art in diffusion models, variational autoencoders, video generation, and controllable synthesis. Mathematical formulations of the key concepts are provided.

\textbf{Chapter~\ref{ch:methodology} -- Methodology} describes the dataset, evaluation metrics, the proposed Semantic VAE architecture, the two-stage diffusion pipeline, and the loss functions used for training.

\textbf{Chapter~\ref{ch:experiments} -- Experiments} presents the implementation details, hyperparameter configurations, quantitative results, ablation studies, and comparisons with baseline methods.

\textbf{Chapter~\ref{ch:discussion} -- Discussion} analyzes the experimental results, discusses training instabilities, limitations of evaluation metrics, and compares the proposed approach with related work.

\textbf{Chapter~\ref{ch:summary} -- Summary} concludes the thesis and outlines directions for future work.

% =================================================================
% CHAPTER 2: RELATED WORK
% =================================================================
\chapter{Related Work}
\label{ch:related_work}

This chapter provides the theoretical background and reviews the state of the art in the areas relevant to this thesis: diffusion models, variational autoencoders, video generation, and controllable synthesis.

% -----------------------------------------------------------------
\section{Continuous-Time Generative Modeling: From Flows to Diffusion}
\label{sec:diffusion_models}

Modern generative models such as diffusion models and flow matching methods can be understood from a unified perspective: they construct a generative process by simulating a differential equation that transforms a simple distribution into a complex data distribution. This section presents a continuous-time formulation based on ordinary and stochastic differential equations, providing a principled foundation for both flow models and diffusion models.

\subsection{Generative Modeling as Distribution Transport}
\label{subsec:distribution_transport}

Let $p_{\text{data}}(\mathbf{x})$ denote the data distribution over $\mathbb{R}^d$, and let $p_0(\mathbf{x})$ denote a simple prior distribution, typically chosen as a standard Gaussian:
\begin{align}
    p_0(\mathbf{x}) = \mathcal{N}(\mathbf{0}, \mathbf{I}).
    \label{eq:prior}
\end{align}
The objective of generative modeling is to construct a transformation that maps samples from $p_0$ to samples from $p_{\text{data}}$.

A natural way to formalize this transformation is to define a \emph{probability path}
\begin{align}
    \{p_t(\mathbf{x})\}_{t \in [0,1]}
    \label{eq:prob_path}
\end{align}
such that
\begin{align}
    p_0(\mathbf{x}) = \mathcal{N}(\mathbf{0}, \mathbf{I}), \qquad p_1(\mathbf{x}) = p_{\text{data}}(\mathbf{x}).
    \label{eq:path_boundary}
\end{align}
If we can construct dynamics that evolve samples according to this path, we obtain a continuous-time generative model. The two main approaches to constructing such dynamics---ordinary differential equations and stochastic differential equations---form the mathematical foundation of modern generative modeling~\cite{song2021scorebased, lipman2023flow, chen2018neural}.

\subsection{Deterministic Flows and Ordinary Differential Equations}
\label{subsec:ode_flows}

We first consider deterministic dynamics defined by an Ordinary Differential Equation (ODE):
\begin{align}
    \frac{d\mathbf{x}_t}{dt} = \mathbf{v}(\mathbf{x}_t, t),
    \label{eq:flow_ode}
\end{align}
where $\mathbf{x}_t \in \mathbb{R}^d$ denotes the state at time $t$, and $\mathbf{v}(\mathbf{x}, t)$ is a time-dependent \emph{vector field} that assigns an instantaneous velocity to every point in space and time. Given an initial sample $\mathbf{x}_0 \sim p_0$, solving the ODE forward in time transports it along a trajectory toward the data distribution.

\paragraph{Evolution of the Probability Density.}
If samples evolve according to the ODE above, the corresponding probability density $p_t(\mathbf{x})$ satisfies the \emph{continuity equation}:
\begin{align}
    \frac{\partial p_t(\mathbf{x})}{\partial t} = -\nabla \cdot \bigl(p_t(\mathbf{x})\,\mathbf{v}(\mathbf{x}, t)\bigr).
    \label{eq:continuity}
\end{align}
This equation expresses conservation of probability mass: the divergence term $\nabla \cdot \mathbf{v}$ controls local expansion or contraction of probability density. Thus, learning a generative model reduces to learning a vector field $\mathbf{v}$ that induces the desired density evolution from $p_0$ to $p_{\text{data}}$.

Neural ODEs~\cite{chen2018neural} parameterize $\mathbf{v}$ with a neural network and solve Equation~\eqref{eq:flow_ode} using numerical ODE solvers, providing a flexible framework for continuous-time generative modeling.

\subsection{Conditional Probability Paths}
\label{subsec:conditional_paths}

Instead of specifying $\mathbf{v}$ directly, it is often more convenient to first define a \emph{conditional probability path}. The marginal path is obtained by averaging over data samples:
\begin{align}
    p_t(\mathbf{x}) = \int p_t(\mathbf{x} \mid \mathbf{x}_1)\, p_{\text{data}}(\mathbf{x}_1)\, d\mathbf{x}_1,
    \label{eq:marginal_path}
\end{align}
where $\mathbf{x}_1 \sim p_{\text{data}}$ is a clean data sample and $p_t(\mathbf{x} \mid \mathbf{x}_1)$ is a designed interpolation between noise and data. A common choice is a \emph{Gaussian conditional path}:
\begin{align}
    \mathbf{x}_t = \alpha(t)\,\mathbf{x}_1 + \sigma(t)\,\boldsymbol{\epsilon}, \qquad \boldsymbol{\epsilon} \sim \mathcal{N}(\mathbf{0}, \mathbf{I}),
    \label{eq:gaussian_interpolant}
\end{align}
where $\alpha(t)$ and $\sigma(t)$ are smooth scalar functions satisfying the boundary conditions:
\begin{align}
    \alpha(0) = 0,\quad \sigma(0) = 1 \qquad \text{(pure noise at } t=0\text{)}, \notag\\
    \alpha(1) = 1,\quad \sigma(1) = 0 \qquad \text{(clean data at } t=1\text{)}.
    \label{eq:path_conditions}
\end{align}
This construction explicitly defines how data points are gradually corrupted into noise (or equivalently, how noise is gradually shaped into data). Different choices of $\alpha(t)$ and $\sigma(t)$ recover different generative frameworks, as discussed in Section~\ref{subsec:unified_view}.

\subsection{Conditional and Marginal Vector Fields}
\label{subsec:vector_fields}

For a fixed data point $\mathbf{x}_1$, the conditional path of Equation~\eqref{eq:gaussian_interpolant} defines a trajectory in expectation. The corresponding \emph{conditional vector field} is obtained by differentiating:
\begin{align}
    \mathbf{v}^*(\mathbf{x}_t, t \mid \mathbf{x}_1) = \frac{d}{dt}\,\mathbb{E}[\mathbf{x}_t \mid \mathbf{x}_1] = \dot{\alpha}(t)\,\mathbf{x}_1 + \dot{\sigma}(t)\,\boldsymbol{\epsilon},
    \label{eq:cond_vector_field}
\end{align}
where $\dot{\alpha}$ and $\dot{\sigma}$ denote time derivatives. However, during generation we do not know $\mathbf{x}_1$; we only observe $\mathbf{x}_t$. Therefore, the true vector field governing the marginal distribution is obtained by taking the conditional expectation:
\begin{align}
    \mathbf{v}^*(\mathbf{x}_t, t) = \mathbb{E}\!\left[\mathbf{v}^*(\mathbf{x}_t, t \mid \mathbf{x}_1) \;\middle|\; \mathbf{x}_t\right].
    \label{eq:marginal_vector_field}
\end{align}
This \emph{marginal vector field} uniquely determines the ODE that transports the full probability distribution from $p_0$ to $p_{\text{data}}$~\cite{lipman2023flow}.

\subsection{Flow Matching Objective}
\label{subsec:flow_matching}

Since the true marginal vector field $\mathbf{v}^*(\mathbf{x}, t)$ is intractable in closed form, we approximate it with a neural network $\mathbf{v}_\theta(\mathbf{x}, t)$. Lipman et al.~\cite{lipman2023flow} showed that the \emph{flow matching} objective can be formulated using only the conditional vector field:
\begin{align}
    \mathcal{L}_{\text{FM}} = \mathbb{E}_{t,\, \mathbf{x}_1,\, \mathbf{x}_t}\!\left[\left\|\mathbf{v}_\theta(\mathbf{x}_t, t) - \mathbf{v}^*(\mathbf{x}_t, t \mid \mathbf{x}_1)\right\|^2\right],
    \label{eq:flow_matching}
\end{align}
where $t \sim \mathcal{U}(0, 1)$, $\mathbf{x}_1 \sim p_{\text{data}}$, and $\mathbf{x}_t \sim p_t(\mathbf{x} \mid \mathbf{x}_1)$. Crucially, the conditional vector field $\mathbf{v}^*(\mathbf{x}_t, t \mid \mathbf{x}_1)$ is known in closed form (Equation~\ref{eq:cond_vector_field}), making the objective tractable.

Training proceeds by:
\begin{enumerate}
    \item Sampling a data point $\mathbf{x}_1 \sim p_{\text{data}}$ and noise $\boldsymbol{\epsilon} \sim \mathcal{N}(\mathbf{0}, \mathbf{I})$,
    \item Sampling a time step $t \sim \mathcal{U}(0, 1)$,
    \item Constructing $\mathbf{x}_t = \alpha(t)\,\mathbf{x}_1 + \sigma(t)\,\boldsymbol{\epsilon}$,
    \item Regressing the neural network output toward the conditional velocity $\mathbf{v}^*(\mathbf{x}_t, t \mid \mathbf{x}_1)$.
\end{enumerate}
This reduces generative modeling to \emph{supervised regression on vector fields}. For the special case of \emph{rectified flow}~\cite{liu2023flow}, where $\alpha(t) = t$ and $\sigma(t) = 1 - t$, the conditional vector field simplifies to $\mathbf{v}^* = \mathbf{x}_1 - \boldsymbol{\epsilon}$, yielding straight-line interpolation paths between noise and data.

\subsection{Stochastic Differential Equations}
\label{subsec:sde}

Diffusion models generalize the deterministic flow framework by introducing stochasticity. Instead of an ODE, the forward process is described by a Stochastic Differential Equation (SDE)~\cite{song2021scorebased}:
\begin{align}
    d\mathbf{x} = \mathbf{f}(\mathbf{x}, t)\, dt + g(t)\, d\mathbf{w}_t,
    \label{eq:forward_sde}
\end{align}
where $\mathbf{f}(\mathbf{x}, t)$ is the \emph{drift} coefficient, $g(t)$ is the \emph{diffusion} coefficient controlling the noise magnitude, and $d\mathbf{w}_t$ denotes standard Brownian motion. The drift determines the deterministic trajectory of the process, while the diffusion term injects random perturbations at each infinitesimal time step.

The corresponding evolution of the probability density is governed by the \emph{Fokker--Planck equation}:
\begin{align}
    \frac{\partial p_t}{\partial t} = -\nabla \cdot (\mathbf{f}\, p_t) + \frac{1}{2}\,g(t)^2\,\Delta p_t,
    \label{eq:fokker_planck}
\end{align}
where $\Delta = \nabla \cdot \nabla$ is the Laplacian operator. Compared to the continuity equation~\eqref{eq:continuity}, the additional term $\frac{1}{2}g(t)^2 \Delta p_t$ introduces stochastic spreading of probability mass. This diffusion term is what gives diffusion models their name.

\subsection{Reverse-Time SDE and Score Functions}
\label{subsec:reverse_sde}

A fundamental result by Anderson~\cite{anderson1982reverse} shows that the reverse-time dynamics of the forward SDE~\eqref{eq:forward_sde} is itself an SDE:
\begin{align}
    d\mathbf{x} = \left[\mathbf{f}(\mathbf{x}, t) - g(t)^2\,\nabla_{\mathbf{x}} \log p_t(\mathbf{x})\right] dt + g(t)\, d\bar{\mathbf{w}}_t,
    \label{eq:reverse_sde}
\end{align}
where $d\bar{\mathbf{w}}_t$ denotes reverse-time Brownian motion. The critical quantity
\begin{align}
    \mathbf{s}(\mathbf{x}, t) \coloneqq \nabla_{\mathbf{x}} \log p_t(\mathbf{x})
    \label{eq:score_function}
\end{align}
is called the \emph{score function}---the gradient of the log-probability density at time $t$. The score points in the direction of increasing probability and its magnitude indicates how steeply the density changes. Intuitively, the reverse SDE follows the drift of the forward process but adds a correction term proportional to the score that ``steers'' samples toward regions of higher data probability.

Diffusion models approximate the score using a neural network $\mathbf{s}_\theta(\mathbf{x}, t) \approx \nabla_{\mathbf{x}} \log p_t(\mathbf{x})$, trained via \emph{denoising score matching}~\cite{song2021scorebased}:
\begin{align}
    \mathcal{L}_{\text{SM}} = \mathbb{E}_{t,\, \mathbf{x}_0,\, \mathbf{x}_t}\!\left[\left\|\mathbf{s}_\theta(\mathbf{x}_t, t) - \nabla_{\mathbf{x}_t} \log p_t(\mathbf{x}_t \mid \mathbf{x}_0)\right\|^2\right].
    \label{eq:score_matching}
\end{align}
For Gaussian conditional paths, $\nabla_{\mathbf{x}_t} \log p_t(\mathbf{x}_t \mid \mathbf{x}_0) = -\boldsymbol{\epsilon} / \sigma(t)$, and this objective becomes equivalent to the noise-prediction loss used in DDPMs (see Section~\ref{subsec:ddpm}).

\paragraph{Probability Flow ODE.}
Song et al.~\cite{song2021scorebased} further showed that any diffusion SDE has a corresponding deterministic ODE---the \emph{probability flow ODE}---that induces the same marginal densities $p_t(\mathbf{x})$:
\begin{align}
    \frac{d\mathbf{x}}{dt} = \mathbf{f}(\mathbf{x}, t) - \frac{1}{2}\,g(t)^2\,\nabla_{\mathbf{x}} \log p_t(\mathbf{x}).
    \label{eq:prob_flow_ode}
\end{align}
This ODE enables deterministic sampling with numerical ODE solvers (e.g., Euler, Heun, or adaptive-step methods), offering a trade-off between stochastic diversity and deterministic speed. It also enables exact likelihood computation via the instantaneous change-of-variables formula.

\subsection{Conditional Generation and Guidance}
\label{subsec:cfg}

In many applications, we wish to condition generation on an external variable $\mathbf{c}$ (e.g., text descriptions, bounding boxes, semantic maps, or trajectories). The conditional score decomposes via Bayes' rule:
\begin{align}
    \nabla_{\mathbf{x}} \log p_t(\mathbf{x} \mid \mathbf{c}) = \nabla_{\mathbf{x}} \log p_t(\mathbf{x}) + \nabla_{\mathbf{x}} \log p(\mathbf{c} \mid \mathbf{x}).
    \label{eq:conditional_score}
\end{align}
\emph{Classifier guidance}~\cite{dhariwal2021diffusion} approximates the second term using an auxiliary classifier trained on noisy data. However, this requires training a separate classifier and may introduce artifacts.

\emph{Classifier-free guidance} (CFG)~\cite{ho2022classifierfree} avoids this by jointly training conditional and unconditional score estimates within a single model. During training, the conditioning $\mathbf{c}$ is randomly dropped with some probability, so the model learns both $\mathbf{s}_\theta(\mathbf{x}, t, \mathbf{c})$ and $\mathbf{s}_\theta(\mathbf{x}, t, \varnothing)$. During inference, the two estimates are combined:
\begin{align}
    \tilde{\mathbf{s}}_\theta(\mathbf{x}, t, \mathbf{c}) = (1 + w)\,\mathbf{s}_\theta(\mathbf{x}, t, \mathbf{c}) - w\,\mathbf{s}_\theta(\mathbf{x}, t, \varnothing),
    \label{eq:cfg}
\end{align}
where $w > 0$ is the \emph{guidance scale}. Higher values of $w$ increase adherence to the condition at the expense of sample diversity. CFG is widely used in modern diffusion models including Stable Diffusion and Stable Video Diffusion, and is employed in both stages of the pipeline presented in this thesis.

\subsection{Unified View}
\label{subsec:unified_view}

Both flow matching and diffusion models can be understood within a unified differential equation framework~\cite{lipman2023flow, song2021scorebased, albergo2023stochastic}:

\begin{itemize}
    \item \textbf{Flow models} learn deterministic vector fields $\mathbf{v}_\theta$ and generate samples by solving ODEs (Equation~\ref{eq:flow_ode}).
    \item \textbf{Diffusion models} learn score functions $\mathbf{s}_\theta$ governing reverse-time SDEs (Equation~\ref{eq:reverse_sde}) or their equivalent probability flow ODEs (Equation~\ref{eq:prob_flow_ode}).
    \item \textbf{Discrete diffusion models} (e.g., DDPM~\cite{ho2020denoising}) arise from time-discretization of continuous SDEs with specific choices of $\mathbf{f}$ and $g$.
\end{itemize}

The connection between flow matching and diffusion is made precise by noting that the score function and the marginal vector field are related via:
\begin{align}
    \mathbf{v}^*(\mathbf{x}_t, t) = \mathbf{f}(\mathbf{x}_t, t) - \frac{1}{2}\,g(t)^2\,\nabla_{\mathbf{x}} \log p_t(\mathbf{x}_t),
    \label{eq:flow_score_relation}
\end{align}
which is exactly the probability flow ODE~\eqref{eq:prob_flow_ode}. This perspective provides a coherent mathematical foundation for modern generative modeling and naturally extends to conditional and controllable generation settings, as exploited throughout this thesis.

\subsection{Denoising Diffusion Probabilistic Models}
\label{subsec:ddpm}

Denoising Diffusion Probabilistic Models (DDPMs)~\cite{ho2020denoising, sohldickstein2015deep} provide a concrete, discrete-time instantiation of the continuous framework presented above. The forward process is defined by a Markov chain with Gaussian transitions:
\begin{align}
    q(\mathbf{x}_t \mid \mathbf{x}_{t-1}) = \mathcal{N}\!\left(\mathbf{x}_t;\, \sqrt{1 - \beta_t}\,\mathbf{x}_{t-1},\, \beta_t\,\mathbf{I}\right),
    \label{eq:forward_step}
\end{align}
where $\{\beta_t\}_{t=1}^T$ is a variance schedule. Defining $\alpha_t = 1 - \beta_t$ and $\bar{\alpha}_t = \prod_{s=1}^t \alpha_s$, the noisy sample at any step $t$ admits a closed-form expression:
\begin{align}
    q(\mathbf{x}_t \mid \mathbf{x}_0) = \mathcal{N}\!\left(\mathbf{x}_t;\, \sqrt{\bar{\alpha}_t}\,\mathbf{x}_0,\, (1 - \bar{\alpha}_t)\,\mathbf{I}\right),
    \label{eq:forward_closed}
\end{align}
which corresponds to the continuous Gaussian path~\eqref{eq:gaussian_interpolant} with $\alpha(t) = \sqrt{\bar{\alpha}_t}$ and $\sigma(t) = \sqrt{1 - \bar{\alpha}_t}$.

The reverse process is parameterized as:
\begin{align}
    p_\theta(\mathbf{x}_{t-1} \mid \mathbf{x}_t) = \mathcal{N}\!\left(\mathbf{x}_{t-1};\, \boldsymbol{\mu}_\theta(\mathbf{x}_t, t),\, \sigma_t^2\,\mathbf{I}\right).
    \label{eq:reverse}
\end{align}
Ho et al.~\cite{ho2020denoising} showed that parameterizing the model to predict the noise $\boldsymbol{\epsilon}_\theta(\mathbf{x}_t, t)$ rather than the mean directly leads to a simplified training objective:
\begin{align}
    \mathcal{L}_{\text{simple}} = \mathbb{E}_{t,\, \mathbf{x}_0,\, \boldsymbol{\epsilon}}\!\left[\left\|\boldsymbol{\epsilon} - \boldsymbol{\epsilon}_\theta(\mathbf{x}_t, t)\right\|^2\right],
    \label{eq:ddpm_loss}
\end{align}
which is equivalent to the score matching objective~\eqref{eq:score_matching} up to a time-dependent weighting factor, since $\boldsymbol{\epsilon} = -\sigma(t)\,\nabla_{\mathbf{x}_t} \log p_t(\mathbf{x}_t \mid \mathbf{x}_0)$. This connection confirms that DDPM training is a discretized form of score function learning.

Subsequent improvements by Nichol and Dhariwal~\cite{nichol2021improved} introduced learned variance schedules and importance-weighted training, while Dhariwal and Nichol~\cite{dhariwal2021diffusion} demonstrated that diffusion models surpass GANs~\cite{goodfellow2014generative} in image generation quality when combined with classifier guidance.

% -----------------------------------------------------------------
\section{Latent Diffusion Models}
\label{sec:ldm}

\subsection{Motivation and Architecture}
\label{subsec:ldm_motivation}

Applying diffusion directly in pixel space is computationally expensive, particularly for high-resolution images and video sequences. The cost scales with spatial resolution, as the denoising network must operate on full-resolution tensors at every diffusion step.

Rombach et al.~\cite{rombach2022highresolution} introduced \emph{Latent Diffusion Models} (LDMs), which perform diffusion in a learned latent space rather than in pixel space. The key idea is to separate:
\begin{itemize}
    \item Perceptual compression, and
    \item Generative modeling.
\end{itemize}

\paragraph{Perceptual Compression via VAE.}
A variational autoencoder (VAE) is first trained to encode images into a lower-dimensional latent representation. Let $\mathbf{x} \in \mathbb{R}^{H \times W \times 3}$ denote an image. The encoder $\mathcal{E}$ maps it to a latent tensor:
\begin{align}
    \mathbf{z} = \mathcal{E}(\mathbf{x}) \in \mathbb{R}^{h \times w \times c},
    \label{eq:vae_encode}
\end{align}
where $h = H/f$, $w = W/f$, and $f$ is a spatial downsampling factor (typically $f=8$). The decoder $\mathcal{D}$ reconstructs the image:
\begin{align}
    \hat{\mathbf{x}} = \mathcal{D}(\mathbf{z}),
    \label{eq:vae_decode}
\end{align}
such that $\mathcal{D}(\mathcal{E}(\mathbf{x})) \approx \mathbf{x}$. The VAE is trained with a combination of reconstruction and perceptual losses to ensure that the latent representation preserves semantically relevant structure.

By operating in latent space, the dimensionality is reduced by a factor of $f^2$, significantly lowering computational cost.

\paragraph{Diffusion in Latent Space.}
After training the VAE, diffusion is performed on latent variables:
\begin{align}
    \mathbf{z}_0 = \mathcal{E}(\mathbf{x}).
    \label{eq:latent_init}
\end{align}
The forward and reverse diffusion processes are then applied to $\mathbf{z}$ rather than $\mathbf{x}$:
\begin{align}
    q(\mathbf{z}_t \mid \mathbf{z}_0), \quad p_\theta(\mathbf{z}_{t-1} \mid \mathbf{z}_t).
    \label{eq:latent_diffusion}
\end{align}
The training objective remains identical to pixel-space diffusion (e.g., noise prediction or score matching as in Equations~\eqref{eq:ddpm_loss} and~\eqref{eq:score_matching}), but the model now learns to denoise latent representations instead of raw pixels.

At inference time:
\begin{enumerate}
    \item Sample $\mathbf{z}_T \sim \mathcal{N}(\mathbf{0}, \mathbf{I})$,
    \item Iteratively denoise to obtain $\mathbf{z}_0$,
    \item Decode using $\mathcal{D}$ to produce the final image.
\end{enumerate}

\paragraph{Conditioning via Cross-Attention.}
The denoising network in LDMs is typically a UNet~\cite{ronneberger2015unet} augmented with cross-attention layers~\cite{vaswani2017attention} to incorporate conditioning signals such as text embeddings, class labels, or spatial maps. Given intermediate UNet features projected to queries $\mathbf{Q}$, and conditioning embeddings projected to keys $\mathbf{K}$ and values $\mathbf{V}$, cross-attention is computed as:
\begin{align}
    \text{Attention}(\mathbf{Q}, \mathbf{K}, \mathbf{V}) = \text{softmax}\!\left(\frac{\mathbf{Q}\mathbf{K}^\top}{\sqrt{d}}\right)\mathbf{V},
    \label{eq:cross_attention}
\end{align}
where $d$ is the dimensionality of the attention head. This mechanism allows the generative process to be guided by external information while maintaining spatial consistency in the latent representation.

\paragraph{Relevance to Video Diffusion.}
Latent diffusion significantly reduces computational cost and memory usage, enabling extension to high-resolution images and spatio-temporal video models. Stable Diffusion~\cite{rombach2022highresolution} and its video extensions build upon this architecture, and the Ctrl-V framework adopts the same latent-space diffusion backbone. Operating in latent space is therefore essential for scalable controllable video generation.

\subsection{EDM Noise Scheduling}
\label{subsec:edm}

In the continuous-time formulation presented above, diffusion models are defined through a stochastic differential equation (SDE) that gradually perturbs data toward Gaussian noise. In practice, early diffusion models such as DDPM discretized this process using a predefined variance schedule $\{\beta_t\}$ indexed by time $t$.

While effective, this time-based parameterization introduces several limitations:
\begin{itemize}
    \item The noise level is indirectly controlled by discrete time indices,
    \item The dynamic range of signal-to-noise ratios varies significantly across timesteps,
    \item Training stability depends heavily on the choice of variance schedule,
    \item Sampling procedures are tightly coupled to the discretization design.
\end{itemize}

To address these issues, Karras et al.~\cite{karras2022elucidating} proposed the EDM (Elucidating the Design Space of Diffusion Models) framework, which reformulates diffusion directly in terms of a continuous noise scale $\sigma$.

\paragraph{Continuous Noise Parameterization.}
Instead of indexing the forward process by time $t$, EDM defines corruption using an explicit noise magnitude:
\begin{align}
    \mathbf{x}_\sigma = \mathbf{x}_0 + \sigma\,\boldsymbol{\epsilon}, \quad \boldsymbol{\epsilon} \sim \mathcal{N}(\mathbf{0}, \mathbf{I}),
    \label{eq:edm_noise}
\end{align}
where $\mathbf{x}_0$ is a clean sample and $\sigma > 0$ directly controls the signal-to-noise ratio. This formulation can be interpreted as a variance-exploding SDE discretized by noise scale rather than time. Small $\sigma$ corresponds to near-clean samples, while large $\sigma$ approximates pure noise.

By using $\sigma$ as the primary parameter, the diffusion process becomes independent of a specific time discretization. This provides greater flexibility in both training and sampling.

\paragraph{Preconditioned Denoiser.}
Another key insight of EDM is that a single neural network must operate across a wide range of noise levels. Without appropriate scaling, the network faces unstable gradients and poorly conditioned inputs. EDM therefore introduces a preconditioned denoiser:
\begin{align}
    D_\theta(\mathbf{x}_\sigma; \sigma) = c_{\text{skip}}(\sigma)\,\mathbf{x}_\sigma + c_{\text{out}}(\sigma)\,F_\theta\!\left(c_{\text{in}}(\sigma)\,\mathbf{x}_\sigma;\, c_{\text{noise}}(\sigma)\right),
    \label{eq:edm_denoiser}
\end{align}
where the scaling functions normalize the input and output across different noise magnitudes. This design ensures:
\begin{itemize}
    \item Stable gradients across noise scales,
    \item Improved numerical conditioning,
    \item Decoupling between noise scheduling and network architecture.
\end{itemize}

\paragraph{Relation to the Continuous Diffusion Framework.}
From the perspective of Section~\ref{sec:diffusion_models}, EDM can be viewed as:
\begin{itemize}
    \item Choosing a specific probability path $p_\sigma(\mathbf{x})$,
    \item Learning the reverse-time dynamics using a preconditioned score or denoiser,
    \item Parameterizing the process in terms of noise level instead of abstract time.
\end{itemize}

Modern large-scale diffusion systems, including Stable Diffusion and Stable Video Diffusion~\cite{blattmann2023stable}, adopt this formulation. The Ctrl-V framework~\cite{luo2025ctrlv} used in this work inherits the EDM noise parameterization, making it directly compatible with continuous-time diffusion theory.

% -----------------------------------------------------------------
\section{Variational Autoencoders}
\label{sec:vae}

\subsection{Theoretical Foundation}
\label{subsec:vae_theory}

The Variational Autoencoder (VAE), introduced independently by Kingma and Welling~\cite{kingma2014autoencoding} and Rezende et al.~\cite{rezende2014stochastic}, is a generative model that learns a latent representation $\mathbf{z}$ of data $\mathbf{x}$ by maximizing a variational lower bound on the log-likelihood:
\begin{align}
    \log p(\mathbf{x}) \geq \mathbb{E}_{q_\phi(\mathbf{z}|\mathbf{x})}\!\left[\log p_\theta(\mathbf{x}|\mathbf{z})\right] - D_{\text{KL}}\!\left(q_\phi(\mathbf{z}|\mathbf{x}) \| p(\mathbf{z})\right) = \mathcal{L}_{\text{ELBO}},
    \label{eq:elbo}
\end{align}
where $q_\phi(\mathbf{z}|\mathbf{x}) = \mathcal{N}(\boldsymbol{\mu}_\phi(\mathbf{x}), \boldsymbol{\sigma}_\phi^2(\mathbf{x})\,\mathbf{I})$ is the approximate posterior (encoder), $p_\theta(\mathbf{x}|\mathbf{z})$ is the likelihood (decoder), and $p(\mathbf{z}) = \mathcal{N}(\mathbf{0}, \mathbf{I})$ is the prior. Training maximizes $\mathcal{L}_{\text{ELBO}}$ with respect to both encoder parameters $\phi$ and decoder parameters $\theta$.

\paragraph{Reparameterization Trick.}
To enable backpropagation through the stochastic sampling of $\mathbf{z}$, Kingma and Welling introduced the reparameterization trick:
\begin{align}
    \mathbf{z} = \boldsymbol{\mu}_\phi(\mathbf{x}) + \boldsymbol{\sigma}_\phi(\mathbf{x}) \odot \boldsymbol{\epsilon}, \quad \boldsymbol{\epsilon} \sim \mathcal{N}(\mathbf{0}, \mathbf{I}),
    \label{eq:reparam}
\end{align}
which expresses $\mathbf{z}$ as a deterministic function of $\phi$ and the random variable $\boldsymbol{\epsilon}$.

\paragraph{KL Divergence.}
For a Gaussian approximate posterior and a standard Gaussian prior, the KL divergence admits a closed-form expression:
\begin{align}
    D_{\text{KL}}\!\left(q_\phi(\mathbf{z}|\mathbf{x}) \| p(\mathbf{z})\right) = -\frac{1}{2}\sum_{j=1}^{J}\left(1 + \log \sigma_j^2 - \mu_j^2 - \sigma_j^2\right),
    \label{eq:kl_closed}
\end{align}
where $J$ is the dimensionality of the latent space. This term acts as a regularizer, encouraging the approximate posterior to stay close to the prior and ensuring a smooth, well-structured latent space.

\subsection{VAE in Latent Diffusion Models}
\label{subsec:vae_in_ldm}

In the LDM framework~\cite{rombach2022highresolution}, the VAE serves as a perceptual compression model that maps high-dimensional images into a lower-dimensional latent space. The VAE used in Stable Diffusion employs a KL-regularized autoencoder trained with a combination of:
\begin{itemize}
    \item Pixel-level reconstruction loss ($L_1$ or $L_2$),
    \item Perceptual loss (LPIPS)~\cite{zhang2018unreasonable},
    \item Adversarial loss (patch-based discriminator),
    \item KL divergence regularization.
\end{itemize}

The encoder produces latent codes $\mathbf{z} \in \mathbb{R}^{h \times w \times 4}$ with a spatial downsampling factor of 8, and the decoder reconstructs images from these latents. Crucially, the VAE is trained once and then frozen during diffusion model training, which means all diffusion operations occur in the fixed latent space.

\subsection{3D VAE for Video}
\label{subsec:3d_vae}

Stable Video Diffusion (SVD)~\cite{blattmann2023stable} extends the image VAE to handle video sequences. The architecture uses a \textbf{2D encoder} that processes each frame independently, producing per-frame latents $\mathbf{z}^{(i)} = \mathcal{E}(\mathbf{f}^{(i)})$, and a \textbf{3D temporal decoder} that jointly decodes all frame latents while maintaining temporal coherence through 3D convolutions:
\begin{align}
    \mathcal{E}&: \mathbb{R}^{3 \times H \times W} \rightarrow \mathbb{R}^{4 \times \frac{H}{8} \times \frac{W}{8}}, \quad \text{(per-frame, 2D)} \label{eq:svd_enc}\\
    \mathcal{D}&: \mathbb{R}^{T \times 4 \times \frac{H}{8} \times \frac{W}{8}} \rightarrow \mathbb{R}^{T \times 3 \times H \times W}. \quad \text{(temporal, 3D)} \label{eq:svd_dec}
\end{align}

This asymmetric design---2D encoder, 3D decoder---is a key architectural choice that enables efficient encoding while ensuring temporal smoothness in decoded video sequences. The encoder consists of residual blocks with downsampling, a mid-block with self-attention, and final convolutional layers that map from 128 internal channels to $2 \times 4 = 8$ output channels (mean and log-variance for 4 latent channels). The decoder mirrors this structure with upsampling blocks but additionally includes temporal convolutions for inter-frame coherence.

\subsection{Semantic VAE}
\label{subsec:semantic_vae}

A central challenge addressed in this thesis is encoding discrete semantic segmentation maps through a VAE designed for continuous RGB images. Semantic maps consist of integer class labels (e.g., 0--18 for KITTI-360), which fundamentally differ from the smooth, continuous pixel intensities that the pretrained VAE expects.

\paragraph{The RGB Bottleneck Problem.}
A naive approach---converting semantic labels to an RGB color palette and passing through the standard VAE---suffers from an information bottleneck: 19 semantic classes must be compressed into 3 RGB channels. The VAE's learned smooth latent manifold further degrades sharp class boundaries, as the model was trained to reconstruct natural images with smooth gradients rather than discrete label maps. Experiments in Section~\ref{sec:vae_results} show that this approach achieves only 54.3\% mIoU despite 97.7\% pixel accuracy, revealing that boundaries account for a small fraction of pixels but a large fraction of errors.

\paragraph{Semantic-Native Architecture.}
To overcome this limitation, we propose a Semantic-Native VAE that bypasses the RGB input/output layers of the pretrained VAE. Instead of feeding 3-channel RGB inputs, the model introduces:
\begin{itemize}
    \item A \textbf{Semantic Stem} (trainable): Maps one-hot encoded semantic labels ($C=19$ channels) to the 128-channel feature space expected by the VAE encoder core.
    \item A \textbf{Semantic Head} (trainable): Maps the 128-channel decoder output back to $C=19$ class logits, using 3D convolutions for temporal consistency.
\end{itemize}

The frozen VAE encoder and decoder cores---comprising all layers between the input convolution and output convolution---are retained unchanged. This allows the model to leverage the powerful spatial representations learned during pretraining while adapting the input and output interfaces to the semantic domain.

% -----------------------------------------------------------------
\section{Video Diffusion Models}
\label{sec:video_diffusion}

\subsection{From Image to Video Diffusion}
\label{subsec:image_to_video}

Extending diffusion models from images to videos introduces the challenge of modeling temporal dynamics in addition to spatial content. Several strategies have been proposed:

\paragraph{Temporal Layers.}
Blattmann et al.~\cite{blattmann2023align} proposed inserting temporal attention and temporal convolution layers into a pretrained 2D UNet, then finetuning only these new layers on video data. This ``inflate-and-finetune'' approach preserves the spatial knowledge of the image model while learning temporal relationships.

\paragraph{Joint Space-Time Modeling.}
Ho et al.~\cite{ho2022video} proposed Video Diffusion Models that perform diffusion jointly over all frames using 3D UNet architectures, treating the video as a 4D tensor $\mathbf{x} \in \mathbb{R}^{T \times C \times H \times W}$.

\paragraph{Cascaded Models.}
Make-A-Video~\cite{singer2023makeavideo} decomposes the problem into separate spatial and temporal components, first generating a key frame and then interpolating or extrapolating to produce the full video sequence.

\subsection{Stable Video Diffusion}
\label{subsec:svd}

Stable Video Diffusion (SVD)~\cite{blattmann2023stable} is a latent video diffusion model that generates video sequences from a single conditioning image. Key design choices include:

\begin{itemize}
    \item \textbf{Architecture:} A UNet denoiser with interleaved spatial and temporal attention layers, operating on latent representations of size $T \times 4 \times H/8 \times W/8$.
    \item \textbf{Conditioning:} The first frame $\mathbf{f}^{(0)}$ is encoded via both the VAE encoder (providing spatial conditioning through concatenation with the noisy latents) and a CLIP image encoder~\cite{radford2021learning} (providing semantic conditioning through cross-attention).
    \item \textbf{Noise Schedule:} EDM-based scheduling~\cite{karras2022elucidating} with continuous noise levels.
    \item \textbf{Training Data Curation:} A systematic pipeline for selecting high-quality video clips with sufficient motion, appropriate aesthetics, and minimal artifacts.
\end{itemize}

SVD operates in image-to-video (I2V) mode: given a single input image, it generates a temporally coherent video continuation. The model variant SVD-XT generates 25 frames at resolutions up to $576 \times 1024$. This pretrained model serves as the backbone for the Ctrl-V framework and, by extension, for the work presented in this thesis.

\begin{figure}[t]
    \centering
    \includegraphics[width=0.9\textwidth]{images/svd_architecture.png}
    \caption{Architecture overview of Stable Video Diffusion. The UNet denoiser operates in latent space and incorporates both spatial and temporal attention layers. The initial frame provides conditioning through both VAE-encoded latents and CLIP embeddings.}
    \label{fig:svd_architecture}
\end{figure}

% -----------------------------------------------------------------
\section{Controllable Video Generation}
\label{sec:controllable}

While diffusion models achieve impressive generative quality, they are inherently stochastic and primarily guided by weak conditioning signals such as text prompts. For applications such as autonomous driving video synthesis, robotics, or simulation, this is insufficient. In these domains, one requires \emph{explicit spatial and structural control}---enforcing object locations, trajectories, or scene layouts with precision.

Controllable diffusion models address this limitation by augmenting pretrained diffusion backbones with mechanisms that inject structured conditioning signals while preserving generative quality. This section reviews the key architectures that enable such control.

\subsection{ControlNet}
\label{subsec:controlnet}

\paragraph{Motivation.}
Pretrained diffusion models such as Stable Diffusion learn powerful image priors from large-scale datasets. However, fine-tuning the entire UNet to incorporate new conditioning signals often leads to:
\begin{itemize}
    \item Loss of pretrained generative quality,
    \item Overfitting on small control datasets,
    \item Catastrophic forgetting of the learned image prior.
\end{itemize}
ControlNet~\cite{zhang2023adding} was proposed to solve this problem by enabling spatial conditioning \emph{without modifying the pretrained backbone weights}.

\paragraph{Core Architecture.}
ControlNet creates a \emph{trainable copy} of the encoder blocks of a pretrained UNet while keeping the original UNet \emph{frozen}. Let:
\begin{itemize}
    \item $\mathcal{F}(\cdot;\, \Theta)$ be the pretrained (frozen) UNet block,
    \item $\mathcal{F}(\cdot;\, \Theta_c)$ be its trainable copy,
    \item $\mathbf{c}$ be the conditioning signal (e.g., edge map, depth map, segmentation mask),
    \item $\mathcal{Z}(\cdot;\, \Theta_z)$ be zero-initialized convolution layers.
\end{itemize}
The modified block output is:
\begin{align}
    \mathbf{y}_c = \mathcal{F}(\mathbf{x};\, \Theta) + \mathcal{Z}\!\left(\mathcal{F}\!\left(\mathbf{x} + \mathcal{Z}(\mathbf{c};\, \Theta_{z1});\, \Theta_c\right);\, \Theta_{z0}\right).
    \label{eq:controlnet}
\end{align}

\paragraph{Step-by-Step Interpretation.}
Equation~\eqref{eq:controlnet} can be understood through four sequential operations:

\begin{enumerate}
    \item \textbf{Baseline generation} --- $\mathcal{F}(\mathbf{x};\, \Theta)$: This is the original pretrained diffusion block. At initialization, the model behaves exactly like the original diffusion model, producing the same output as if no conditioning were present.

    \item \textbf{Conditioning injection} --- $\mathbf{x} + \mathcal{Z}(\mathbf{c};\, \Theta_{z1})$: The conditioning signal $\mathbf{c}$ is passed through a zero-initialized convolution layer. Because all weights are initialized to zero:
    \begin{align}
        \mathcal{Z}(\mathbf{c};\, \Theta_{z1}) = \mathbf{0} \quad \text{at initialization},
        \label{eq:zero_init}
    \end{align}
    and thus $\mathbf{x} + \mathbf{0} = \mathbf{x}$. The conditioning signal does not disturb the pretrained model at the start of training.

    \item \textbf{Trainable copy processing} --- $\mathcal{F}(\mathbf{x} + \mathcal{Z}(\mathbf{c});\, \Theta_c)$: This branch learns to transform the conditioning features into spatially meaningful information. Since $\Theta_c$ is initialized from $\Theta$, the trainable copy starts with the same representational capacity as the pretrained model.

    \item \textbf{Zero-initialized residual injection} --- $\mathcal{Z}(\cdot;\, \Theta_{z0})$: A second zero-initialized convolution ensures that the output of the trainable copy contributes nothing at initialization:
    \begin{align}
        \mathbf{y}_c = \mathcal{F}(\mathbf{x};\, \Theta) + \mathbf{0} = \mathcal{F}(\mathbf{x};\, \Theta) \quad \text{at initialization}.
        \label{eq:controlnet_init}
    \end{align}
\end{enumerate}

As training progresses, the parameters $\Theta_c$ learn meaningful control features, while $\Theta_{z0}$ and $\Theta_{z1}$ gradually allow these features to influence generation. This ensures that training \emph{starts from the pretrained solution} and the control signal is incorporated without destroying the learned prior.

\paragraph{Importance of Zero-Convolution.}
Without zero-initialization, the model output would change immediately upon adding the ControlNet branch, degrading pretrained generative quality from the first training step. Zero-convolutions guarantee that:
\begin{itemize}
    \item The pretrained model is perfectly preserved at initialization,
    \item The conditioning influence grows smoothly during training,
    \item Training is stable and data-efficient, even with small control datasets.
\end{itemize}
This design made ControlNet remarkably robust in practice and contributed to its widespread adoption.

\paragraph{Supported Control Modalities.}
ControlNet supports a wide range of spatial conditioning signals, including Canny edges, depth maps, human pose skeletons, segmentation masks, and scribbles. In practice, it transformed diffusion models from purely artistic generators into \emph{controllable image synthesis systems}, where the user can specify precise spatial structure while the diffusion model fills in realistic appearance and detail.

\paragraph{Extension to Video.}
Originally designed for image diffusion, ControlNet has been adapted to video diffusion models~\cite{hu2023videocontrolnet, chen2023controlavideo}. In the video setting:
\begin{itemize}
    \item Conditioning is applied independently per frame,
    \item The resulting control features are injected into a spatio-temporal UNet,
    \item Temporal attention layers maintain motion consistency across frames.
\end{itemize}
This extension enables spatial control that is coherent across time, which is essential for applications such as driving video generation.

\begin{figure}[t]
    \centering
    \includegraphics[width=0.9\textwidth]{images/controlnet_architecture.png}
    \caption{ControlNet architecture. A trainable copy of the UNet encoder receives the conditioning input and injects its features into the frozen UNet decoder via zero-initialized convolution layers. At initialization, the zero-convolutions ensure the model output is identical to the pretrained model.}
    \label{fig:controlnet_architecture}
\end{figure}

\subsection{Bounding-Box Control: Ctrl-V}
\label{subsec:ctrlv}

\paragraph{Motivation.}
In autonomous driving simulation, control via generic edges or depth maps is insufficient. Instead, the application demands:
\begin{itemize}
    \item Explicit object identities and categories,
    \item Bounding box trajectories specifying object motion,
    \item Temporal consistency across the generated video,
    \item Physical plausibility of object dynamics.
\end{itemize}
Ctrl-V~\cite{luo2025ctrlv} addresses these requirements by conditioning Stable Video Diffusion (SVD) on bounding box trajectories through a two-stage framework.

\paragraph{Stage~1 --- Bounding Box Generator.}
The first stage predicts bounding box trajectories using a diffusion model built on the SVD backbone. Given:
\begin{itemize}
    \item An initial RGB frame $\mathbf{f}^{(0)}$,
    \item An initial bounding box frame $\mathbf{b}^{(0)}$,
    \item Optionally, a final bounding box frame $\mathbf{b}^{(N-1)}$,
\end{itemize}
the model generates the full bounding box video sequence:
\begin{align}
    \mathbf{b} = \left[\mathbf{b}^{(0)},\, \mathbf{b}^{(1)},\, \ldots,\, \mathbf{b}^{(N-1)}\right]
    \label{eq:bbox_sequence}
\end{align}
through the diffusion process in latent space. This stage converts sparse control inputs into a complete temporal trajectory, providing dense frame-by-frame conditioning for the subsequent video generation stage.

\paragraph{Stage~2 --- Box2Video Generation.}
The predicted bounding box sequence is encoded via the VAE and injected into SVD through a ControlNet module. The noise prediction combines the frozen SVD backbone with the trainable ControlNet:
\begin{align}
    \boldsymbol{\epsilon}_\theta(\hat{\mathbf{z}}_t, t, \mathbf{z}^{(0)}, \mathbf{c}^{(0)}, \mathbf{b}) = \mathbb{U}_\theta\!\left(\hat{\mathbf{z}}_t, t, \mathbf{z}^{(0)}_{\text{pad}}, \mathbf{c}^{(0)}\right) + \text{ControlNet}_\xi(\hat{\mathbf{z}}_t, \mathbf{b}),
    \label{eq:ctrlv_box2video}
\end{align}
where:
\begin{itemize}
    \item $\mathbb{U}_\theta$ is the frozen SVD UNet backbone,
    \item $\hat{\mathbf{z}}_t$ is the noisy video latent at diffusion step $t$,
    \item $\mathbf{z}^{(0)}_{\text{pad}}$ is the initial frame latent padded along the temporal dimension,
    \item $\mathbf{c}^{(0)}$ is the CLIP embedding of the initial frame,
    \item $\mathbf{b}$ is the VAE-encoded bounding box sequence,
    \item $\xi$ denotes the trainable ControlNet parameters.
\end{itemize}

The key design property is that $\theta$ \emph{remains frozen} while only $\xi$ \emph{is trained}. This preserves the pretrained video generation prior intact; only the control branch adapts to the bounding box conditioning. The bounding box trajectories guide object motion while the frozen SVD backbone ensures high visual quality and temporal coherence.

\paragraph{Advantages of Bounding Box Control.}
Compared to pixel-level conditioning (e.g., dense segmentation maps), bounding boxes provide:
\begin{itemize}
    \item Explicit object-level control with interpretable structure,
    \item Compact representation that scales to long sequences,
    \item Direct compatibility with object detection datasets commonly used in autonomous driving.
\end{itemize}
This makes Ctrl-V particularly suitable for driving simulation, where object trajectories are a natural and available form of supervision.

\subsection{Other Control Modalities}
\label{subsec:other_controls}

Beyond bounding boxes, various spatial and motion control strategies have been explored for video generation:
\begin{itemize}
    \item \textbf{Depth and Edge Maps:} VideoControlNet~\cite{hu2023videocontrolnet} uses per-frame depth, Canny edges, and optical flow as frame-level conditions for video diffusion.
    \item \textbf{Motion Modules:} AnimateDiff~\cite{guo2024animatediff} introduces learnable motion layers that can be conditioned on motion trajectories, enabling temporal dynamics control.
    \item \textbf{Text and Layout:} Boximator~\cite{wang2024boximator} combines textual prompts with hard and soft bounding box constraints for fine-grained motion control.
    \item \textbf{Semantic Layouts:} SPADE-based approaches~\cite{park2019semantic} condition image synthesis on semantic segmentation maps, but their extension to video generation within diffusion frameworks remains underexplored---a gap that this thesis aims to fill.
\end{itemize}

\subsection{Driving-Specific Video Generation}
\label{subsec:driving_generation}

Several recent works target video generation for autonomous driving specifically:
\begin{itemize}
    \item \textbf{GAIA-1}~\cite{hu2023gaia}: A generative world model that produces driving videos from text, action, and video tokens.
    \item \textbf{DrivingDiffusion}~\cite{li2023drivingdiffusion}: Uses layout-guided multi-view generation with latent diffusion.
    \item \textbf{VISTA}~\cite{gao2024vista}: A generalizable driving world model with high fidelity and versatile controllability.
\end{itemize}

These works highlight the growing interest in using generative models for autonomous driving simulation. The approach in this thesis differs by using \emph{dense semantic segmentation maps} as the control signal, providing pixel-level scene specification rather than sparse bounding boxes or text descriptions.

\paragraph{Summary.}
ControlNet introduced a principled mechanism for injecting structured spatial control into pretrained diffusion models without degrading generative quality. Ctrl-V extends this idea to video generation with bounding box trajectories, enabling object-level control and temporal consistency for autonomous driving simulation. These architectures form the foundation for controllable video synthesis and directly motivate the semantic control mechanisms proposed in this thesis.

% =================================================================
% CHAPTER 3: METHODOLOGY
% =================================================================
\chapter{Methodology}
\label{ch:methodology}

This chapter describes the dataset, evaluation metrics, the proposed model architectures, and the loss functions employed in this work.

% -----------------------------------------------------------------
\section{Dataset: KITTI-360}
\label{sec:dataset}

\subsection{Overview}
\label{subsec:dataset_overview}

The KITTI-360 dataset~\cite{liao2022kitti360} is a large-scale outdoor dataset for autonomous driving research, extending the original KITTI benchmark~\cite{geiger2012are} with richer annotations including dense semantic and instance segmentation, 3D bounding boxes, and accumulated point clouds. The dataset was captured in suburban areas of Karlsruhe, Germany, using a Volkswagen station wagon equipped with:
\begin{itemize}
    \item Two high-resolution color cameras (stereo),
    \item A Velodyne HDL-64E 3D laser scanner,
    \item An OXTS RT3003 GPS/IMU localization unit.
\end{itemize}

For this work, we use the front-facing camera (camera~00) images and corresponding semantic segmentation annotations. The dataset is split into 9 driving sequences covering a total of approximately 73,000 frames.

\subsection{Semantic Annotations}
\label{subsec:semantic_annotations}

The semantic annotations use 19 training classes consistent with the Cityscapes label definition~\cite{cordts2016cityscapes}. Raw KITTI-360 label IDs are remapped to continuous training IDs (0--18) using the official \texttt{kitti360scripts} library. Table~\ref{tab:kitti360_classes} lists the semantic classes used.

\begin{table}[t]
    \centering
    \caption{KITTI-360 semantic classes with training IDs used in this work}
    \label{tab:kitti360_classes}
    \begin{tabular}{clcl}
        \toprule
        \textbf{Train ID} & \textbf{Class} & \textbf{Train ID} & \textbf{Class} \\
        \midrule
        0  & Road         & 10 & Sky \\
        1  & Sidewalk     & 11 & Person \\
        2  & Building     & 12 & Rider \\
        3  & Wall         & 13 & Car \\
        4  & Fence        & 14 & Truck \\
        5  & Pole         & 15 & Bus \\
        6  & Traffic Light & 16 & Train \\
        7  & Traffic Sign & 17 & Motorcycle \\
        8  & Vegetation   & 18 & Bicycle \\
        9  & Terrain      &    & \\
        \bottomrule
    \end{tabular}
\end{table}

\subsection{Data Preparation}
\label{subsec:data_preparation}

The official train/validation split files (\texttt{2013\_05\_28\_drive\_train\_frames.txt} and \texttt{2013\_05\_28\_drive\_val\_frames.txt}) define paired RGB and semantic paths. Each pair specifies an RGB image from \texttt{data\_2d\_raw/} and its corresponding semantic label map from \texttt{data\_2d\_semantics/train/}.

Key preprocessing steps:
\begin{itemize}
    \item \textbf{Resolution:} Images are resized from the original $376 \times 1408$ to $192 \times 704$ using bilinear interpolation for RGB and nearest-neighbor interpolation for semantic maps (to preserve discrete labels).
    \item \textbf{Clip Formation:} Consecutive frames are grouped into clips of $T=25$ frames for diffusion training and $T=4$ frames for Semantic VAE training.
    \item \textbf{Normalization:} RGB images are normalized to $[-1, 1]$; semantic labels remain as integer IDs.
\end{itemize}

The training set contains approximately 49,000 frame pairs. For Semantic VAE training, a subset of 500 clips (2,000 frames at $T=4$) is used due to the small number of trainable parameters, with 200 clips reserved for validation.

% -----------------------------------------------------------------
\section{Evaluation Metrics}
\label{sec:metrics}

\subsection{Fr\'echet Inception Distance (FID)}
\label{subsec:fid}

The Fr\'echet Inception Distance (FID)~\cite{heusel2017gans} measures the similarity between the distributions of generated and real images in the feature space of an Inception-v3 network~\cite{szegedy2016rethinking}. Given multivariate Gaussian fits $\mathcal{N}(\boldsymbol{\mu}_r, \boldsymbol{\Sigma}_r)$ and $\mathcal{N}(\boldsymbol{\mu}_g, \boldsymbol{\Sigma}_g)$ to the real and generated feature distributions respectively:
\begin{align}
    \text{FID} = \|\boldsymbol{\mu}_r - \boldsymbol{\mu}_g\|^2 + \text{Tr}\!\left(\boldsymbol{\Sigma}_r + \boldsymbol{\Sigma}_g - 2\left(\boldsymbol{\Sigma}_r \boldsymbol{\Sigma}_g\right)^{1/2}\right).
    \label{eq:fid}
\end{align}
Lower FID scores indicate higher similarity between generated and real image distributions. FID is computed on individual frames extracted from generated and ground-truth videos.

\subsection{Fr\'echet Video Distance (FVD)}
\label{subsec:fvd}

The Fr\'echet Video Distance (FVD)~\cite{unterthiner2019fvd} extends FID to the video domain by using features from an Inflated 3D ConvNet (I3D)~\cite{carreira2017quo} pretrained on Kinetics-400. The I3D network captures both spatial appearance and temporal dynamics:
\begin{align}
    \text{FVD} = \|\boldsymbol{\mu}_r^{\text{I3D}} - \boldsymbol{\mu}_g^{\text{I3D}}\|^2 + \text{Tr}\!\left(\boldsymbol{\Sigma}_r^{\text{I3D}} + \boldsymbol{\Sigma}_g^{\text{I3D}} - 2\left(\boldsymbol{\Sigma}_r^{\text{I3D}} \boldsymbol{\Sigma}_g^{\text{I3D}}\right)^{1/2}\right).
    \label{eq:fvd}
\end{align}
FVD evaluates both the visual quality and temporal coherence of generated videos. Lower values are better.

\subsection{Mean Intersection-over-Union (mIoU)}
\label{subsec:miou}

For evaluating the Semantic VAE, the Mean Intersection-over-Union is the primary metric. For a given class $c$:
\begin{align}
    \text{IoU}_c = \frac{|\mathcal{P}_c \cap \mathcal{T}_c|}{|\mathcal{P}_c \cup \mathcal{T}_c|} = \frac{\text{TP}_c}{\text{TP}_c + \text{FP}_c + \text{FN}_c},
    \label{eq:iou}
\end{align}
where $\mathcal{P}_c$ and $\mathcal{T}_c$ are the sets of pixels predicted and labeled as class $c$, respectively. The mean IoU averages over all valid classes:
\begin{align}
    \text{mIoU} = \frac{1}{|\mathcal{C}_{\text{valid}}|} \sum_{c \in \mathcal{C}_{\text{valid}}} \text{IoU}_c.
    \label{eq:miou}
\end{align}
Unlike pixel accuracy, mIoU treats all classes equally regardless of their spatial prevalence, making it sensitive to performance on rare classes such as poles and traffic signs.

\subsection{Additional Video Quality Metrics}
\label{subsec:additional_metrics}

In addition to FID and FVD, we report several complementary metrics:

\begin{itemize}
    \item \textbf{LPIPS} (Learned Perceptual Image Patch Similarity)~\cite{zhang2018unreasonable}: Measures perceptual distance between image pairs using deep features. Lower is better.
    \item \textbf{SSIM} (Structural Similarity Index): Evaluates structural similarity between generated and ground-truth frames, considering luminance, contrast, and structure. Higher is better, with a maximum of 1.0.
    \item \textbf{PSNR} (Peak Signal-to-Noise Ratio): Measures pixel-level reconstruction quality in decibels. Higher values indicate less distortion.
\end{itemize}

% -----------------------------------------------------------------
\section{Model Architecture}
\label{sec:architecture}

\subsection{Overview of the Two-Stage Pipeline}
\label{subsec:pipeline_overview}

The proposed pipeline, illustrated in Figure~\ref{fig:pipeline_overview}, consists of three components:

\begin{enumerate}
    \item \textbf{Semantic VAE} (pretrained, frozen during diffusion training): Encodes semantic label maps into and decodes them from the latent space shared with the RGB VAE.
    \item \textbf{Stage~1 --- Semantic Prediction Model:} A diffusion model that predicts future semantic video latents given an initial RGB frame.
    \item \textbf{Stage~2 --- Semantic-to-Video Model:} A ControlNet-conditioned diffusion model that generates RGB video latents conditioned on semantic video latents.
\end{enumerate}

\begin{figure}[t]
    \centering
    % TODO: Replace with actual pipeline diagram
    \fbox{\parbox{0.9\textwidth}{\centering\vspace{3cm}\textbf{[Pipeline Overview Diagram]}\\\textit{Two-stage pipeline: Semantic VAE + Stage~1 Semantic Prediction + Stage~2 Semantic-to-Video}\vspace{3cm}}}
    \caption{Overview of the proposed two-stage pipeline for semantically controlled video generation. The Semantic VAE bridges discrete semantic maps and continuous latent space. Stage~1 predicts future semantic sequences. Stage~2 generates photorealistic RGB video conditioned on the predicted semantics.}
    \label{fig:pipeline_overview}
\end{figure}

\subsection{Semantic-Native VAE}
\label{subsec:semantic_native_vae}

The pretrained Stable Video Diffusion (SVD) VAE was trained on natural RGB images and therefore assumes an input tensor $\mathbf{x}_{\text{rgb}} \in \mathbb{R}^{B \times T \times 3 \times H \times W}$. Internally, the VAE begins with a learned projection that converts 3 RGB channels into the feature width expected by the encoder backbone (128 channels), and ends with a symmetric projection from 128-channel decoder features back to 3 RGB channels. However, semantic label maps are \emph{discrete} variables
\begin{align}
    \mathbf{y} \in \{0, \ldots, C-1\}^{B \times T \times H \times W}, \quad C = 19,
    \label{eq:semantic_input}
\end{align}
and do not admit a meaningful representation in 3 channels without information loss. To bridge this mismatch, we introduce a \emph{Semantic-Native VAE} that replaces only the RGB-specific input and output projections while keeping the high-capacity pretrained VAE core frozen. The architecture is illustrated in Figure~\ref{fig:semantic_vae_arch}.

\begin{figure}[t]
    \centering
    \includegraphics[width=0.9\textwidth]{images/semantic_vae_architecture.png}
    \caption{Architecture of the Semantic-Native VAE. The trainable Semantic Stem and Head (shown in blue) replace the original RGB input/output convolutions and interface directly with the frozen VAE encoder and decoder cores (shown in gray), bypassing the RGB bottleneck.}
    \label{fig:semantic_vae_arch}
\end{figure}

\paragraph{Why the Semantic Jump Is Needed: Removing the RGB Bottleneck.}
A naive approach to reuse the RGB VAE for semantic data is to map semantic IDs to a 3-channel color image and feed it into the pretrained encoder. This is problematic for two fundamental reasons:

\begin{enumerate}
    \item \textbf{Information bottleneck ($C \rightarrow 3$):} A one-hot semantic representation contains $C = 19$ mutually exclusive class channels. Mapping this to 3 channels necessarily collapses class separability: different classes may map to similar colors, RGB differences encode appearance rather than class identity, and the mapping is not invertible (many-to-one).

    \item \textbf{Domain mismatch of the learned RGB projection:} The first convolution of the pretrained VAE is trained to extract edges, textures, and colors from natural images. Semantic maps contain piecewise-constant regions and sharp class boundaries with fundamentally different statistics. Feeding such data through an RGB-trained input stem introduces a distribution shift at the earliest layer, which propagates through the entire encoder.
\end{enumerate}

For these reasons, we remove the RGB-specific $3 \rightarrow 128$ projection and replace it with a semantic-specific adapter. This replacement is what we refer to as the \emph{semantic jump}: it directly maps from the semantic domain into the internal feature space of the pretrained VAE core, bypassing the RGB bottleneck entirely. Concretely, the data path changes from:
\begin{align}
    \text{Original:} \quad &\mathbf{x}_{\text{rgb}} \xrightarrow{\text{Conv}^{3 \rightarrow 128}} \text{EncCore} \xrightarrow{} \mathbf{z} \xrightarrow{} \text{DecCore} \xrightarrow{\text{Conv}^{128 \rightarrow 3}} \hat{\mathbf{x}}_{\text{rgb}}, \nonumber \\
    \text{Semantic:} \quad &\mathbf{x}_{\text{oh}} \xrightarrow{\text{Stem}^{C \rightarrow 128}} \text{EncCore} \xrightarrow{} \mathbf{z} \xrightarrow{} \text{DecCore} \xrightarrow{\text{Head}^{128 \rightarrow C}} \hat{\mathbf{s}}.
    \label{eq:path_comparison}
\end{align}

\paragraph{Input Encoding.}
We first convert discrete semantic IDs into a one-hot tensor:
\begin{align}
    \mathbf{x}_{\text{oh}} = \text{OneHot}(\mathbf{y}) \in \{0, 1\}^{BT \times C \times H \times W}.
    \label{eq:onehot}
\end{align}
Pixels with the ignore index (255, used for unlabeled regions) are clamped to class 0 during encoding and excluded from the loss via a binary mask. This representation preserves semantic identity explicitly: each channel corresponds to exactly one class, and no information is lost through color encoding.

\paragraph{Semantic Input Adapter (Replacing the RGB Input Stem).}
In the original RGB VAE, the input projection performs:
$\mathbb{R}^{3 \times H \times W} \rightarrow \mathbb{R}^{128 \times H \times W}.$
We replace this with a trainable two-layer convolutional stem:
\begin{align}
    \text{Stem}(\mathbf{x}_{\text{oh}}) = \text{Conv}_{3 \times 3}^{64 \rightarrow 128}\!\left(\text{SiLU}\!\left(\text{GN}_8\!\left(\text{Conv}_{3 \times 3}^{C \rightarrow 64}(\mathbf{x}_{\text{oh}})\right)\right)\right),
    \label{eq:stem}
\end{align}
where $\text{GN}_8$ denotes Group Normalization~\cite{wu2018group} with 8 groups and $\text{SiLU}(x) = x \cdot \sigma(x)$ is the Sigmoid Linear Unit activation. Both convolutions use $3 \times 3$ kernels with padding~1 and are initialized with Kaiming normal initialization. This adapter serves two purposes: (i)~it removes the semantic compression bottleneck ($C \rightarrow 3$) by never projecting semantics into RGB, and (ii)~it matches the 128-channel interface expected by the pretrained encoder core.

\paragraph{Frozen VAE Encoder Core.}
The output of the semantic stem is fed into the \emph{frozen} pretrained VAE encoder core:
\begin{align}
    \mathbf{h}_{\text{enc}} = \text{EncCore}\!\left(\text{Stem}(\mathbf{x}_{\text{oh}})\right).
    \label{eq:enc_core}
\end{align}
The encoder core processes the 128-channel input through four downsampling blocks with residual connections and self-attention, a mid-block with self-attention, and a final convolution that produces Gaussian posterior parameters $(\boldsymbol{\mu},\, \log \boldsymbol{\sigma}^2) \in \mathbb{R}^{BT \times 8 \times H/8 \times W/8}$. We use \emph{deterministic encoding} for compatibility with the diffusion backbone:
\begin{align}
    \mathbf{z} = \boldsymbol{\mu} \in \mathbb{R}^{BT \times 4 \times H/8 \times W/8}.
    \label{eq:deterministic_latent}
\end{align}
Freezing the encoder core ensures that the resulting latent space remains aligned with the SVD latent geometry used in Stages~1 and~2.

\paragraph{Frozen VAE Decoder Core.}
The frozen decoder core maps latent codes back to 128-channel features through upsampling blocks with temporal (3D) convolutions and self-attention. Each frame is decoded independently with $T_{\text{dec}} = 1$ to produce features:
\begin{align}
    \mathbf{h}_{\text{dec}} = \text{DecCore}(\mathbf{z}) \in \mathbb{R}^{BT \times 128 \times H \times W}.
    \label{eq:dec_features}
\end{align}
Features are captured \emph{before} the original RGB output convolution using forward hooks. This is critical: it allows gradient flow through the frozen decoder back to the trainable stem during backpropagation, while completely bypassing the RGB output layer.

\paragraph{Semantic Output Adapter (Replacing the RGB Output Head).}
The pretrained decoder ends with a projection $\mathbb{R}^{128 \times H \times W} \rightarrow \mathbb{R}^{3 \times H \times W}$ that produces RGB pixel values. This is unsuitable for semantic reconstruction because semantics are \emph{categorical}, not continuous. We replace this with a trainable semantic classifier head:
\begin{align}
    \text{Head}(\mathbf{h}_{\text{dec}}) = \text{Conv}_{1 \times 1 \times 1}^{64 \rightarrow C}\!\left(\text{SiLU}\!\left(\text{GN}_8\!\left(\text{Conv}_{3 \times 3 \times 3}^{128 \rightarrow 64}(\mathbf{h}_{\text{dec}})\right)\right)\right),
    \label{eq:head}
\end{align}
producing per-pixel class logits $\hat{\mathbf{s}} \in \mathbb{R}^{B \times T \times C \times H \times W}$. The use of 3D convolutions ($3 \times 3 \times 3$ kernels in the first layer) allows the head to leverage temporal context for smoother predictions across frames, while the final $1 \times 1 \times 1$ convolution produces class-specific outputs.

\paragraph{Training Strategy: Freeze Core, Train Only Semantic Adapters.}
During training, all pretrained VAE core parameters are frozen:
\begin{align}
    \Theta_{\text{core}} = \{\Theta_{\text{EncCore}},\, \Theta_{\text{DecCore}}\} \quad \text{(frozen)},
    \label{eq:frozen_params}
\end{align}
and only the semantic adapters are optimized:
\begin{align}
    \Theta_{\text{sem}} = \{\Theta_{\text{Stem}},\, \Theta_{\text{Head}}\} \quad \text{(trainable)}.
    \label{eq:trainable_params}
\end{align}
This strategy provides two key benefits:
\begin{enumerate}
    \item \textbf{Latent space preservation:} The pretrained latent geometry used by SVD remains unchanged, ensuring compatibility with the diffusion models in Stages~1 and~2.
    \item \textbf{Parameter efficiency:} By training only $\sim$200K parameters (versus $\sim$84M frozen core parameters), the model avoids overfitting on limited semantic data and converges rapidly.
\end{enumerate}
Although the core is frozen, gradients flow \emph{through} it during backpropagation (it acts as a fixed differentiable mapping), but only $\Theta_{\text{sem}}$ receives parameter updates.

\paragraph{Training Objective: Cross-Entropy Loss.}
The task of the Semantic-Native VAE is to reconstruct a categorical semantic map. Each pixel corresponds to a discrete class index; therefore, this is a \emph{per-pixel classification} problem. The correct likelihood model is categorical:
\begin{align}
    p(y_{i,j} \mid \hat{\mathbf{s}}_{i,j}) = \text{Categorical}\!\left(\text{softmax}(\hat{\mathbf{s}}_{i,j})\right).
    \label{eq:categorical_likelihood}
\end{align}
Maximizing this likelihood leads directly to cross-entropy minimization:
\begin{align}
    \mathcal{L}_{\text{CE}} = -\sum_{b,t,i,j} m_{b,t,i,j} \cdot \log\!\left(\text{softmax}(\hat{\mathbf{s}}_{b,t,:,i,j})\right)_{y_{b,t,i,j}},
    \label{eq:ce_loss}
\end{align}
where $m$ is the ignore mask ($m = 0$ for pixels with ignore index, $m = 1$ otherwise).

Two design choices regarding the loss function merit explicit justification:

\begin{itemize}
    \item \textbf{Why not MSE or regression losses?} Using an $L_2$ loss on class indices would impose an artificial ordering (e.g., ``road = 0'' being numerically closer to ``sidewalk = 1'' than to ``car = 13''), which is semantically meaningless. Cross-entropy is invariant to label index permutations and correctly models discrete, unordered categories.

    \item \textbf{Why no KL regularization term?} A standard VAE objective includes a KL divergence term $\mathcal{L}_{\text{VAE}} = \mathcal{L}_{\text{recon}} + \beta\, D_{\text{KL}}(q(\mathbf{z} \mid \mathbf{x}) \| p(\mathbf{z}))$ to enforce a latent prior. In our case, the encoder core is \emph{frozen} and already trained to produce latents compatible with the diffusion prior. Introducing an additional KL term would either be ineffective (since the core parameters cannot be updated) or would distort the learned semantic adapters away from the pretrained latent geometry. Therefore, we use deterministic latents $\mathbf{z} = \boldsymbol{\mu}$ and optimize only reconstruction via cross-entropy.
\end{itemize}

This design makes the Semantic-Native VAE behave as a \emph{semantic interface to a fixed latent manifold}, rather than a fully retrained generative autoencoder.

\paragraph{Parameter Count.}
The Semantic Stem contains approximately 10,000 parameters and the Semantic Head approximately 190,000 parameters, for a total of $\sim$200,000 trainable parameters. The frozen VAE core contains $\sim$84 million parameters. This parameter-efficient design enables fast convergence with limited training data and minimal risk of overfitting.

\subsection{Stage~1: Semantic Video Prediction}
\label{subsec:stage1}

Stage~1 adapts the SVD backbone to predict semantic video sequences. The model receives:
\begin{itemize}
    \item $\hat{\mathbf{s}}_t \in \mathbb{R}^{T \times 4 \times h \times w}$: Noisy latent representation of the semantic video (being denoised),
    \item $\mathbf{z}^{(0)} \in \mathbb{R}^{1 \times 4 \times h \times w}$: Latent encoding of the initial RGB frame (via the RGB VAE),
    \item $\mathbf{c}^{(0)}$: CLIP encoding of the initial frame (for cross-attention conditioning),
\end{itemize}
where $h = H/8 = 24$ and $w = W/8 = 88$ at the training resolution of $192 \times 704$.

The initial frame latent is padded (repeated) along the temporal dimension and concatenated with the noisy semantic latents along the channel dimension, following the SVD conditioning mechanism. The UNet predicts the noise $\boldsymbol{\epsilon}_\theta(\hat{\mathbf{s}}_t, t, \mathbf{z}^{(0)}_{\text{pad}}, \mathbf{c}^{(0)})$ and is trained with the standard diffusion objective (Equation~\ref{eq:ddpm_loss}).

The encoding of semantic IDs into latent space uses the trained Semantic VAE (frozen during Stage~1 training): semantic IDs $\rightarrow$ Semantic VAE encoder $\rightarrow$ semantic latents $\mathbf{s}$.

\subsection{Stage~2: Semantic-to-Video Generation}
\label{subsec:stage2}

Stage~2 generates photorealistic RGB videos conditioned on semantic maps using a ControlNet architecture~\cite{zhang2023adding} adapted to video. The model consists of:

\begin{itemize}
    \item \textbf{SVD Backbone (frozen):} Receives the initial frame latent $\mathbf{z}^{(0)}_{\text{pad}}$ concatenated with noisy RGB video latents $\hat{\mathbf{z}}_t$, plus CLIP conditioning $\mathbf{c}^{(0)}$ via cross-attention.
    
    \item \textbf{ControlNet (trainable):} A copy of the SVD encoder blocks that receives the same noisy latents plus the semantic latents $\mathbf{s}$ (encoded by the Semantic VAE). The ControlNet output is injected into the SVD decoder via zero-initialized convolutions.
\end{itemize}

The training objective for Stage~2 is:
\begin{align}
    \mathcal{L}_{\text{Stage2}} = \mathbb{E}_{t, \mathbf{z}_0, \boldsymbol{\epsilon}}\!\left[\left\|\boldsymbol{\epsilon} - \boldsymbol{\epsilon}_{\theta,\xi}(\hat{\mathbf{z}}_t, t, \mathbf{z}^{(0)}_{\text{pad}}, \mathbf{c}^{(0)}, \mathbf{s})\right\|^2\right],
    \label{eq:stage2_loss}
\end{align}
where $\theta$ (SVD UNet) is frozen and only $\xi$ (ControlNet) is updated.

During training, ground-truth semantic labels are used as conditioning (encoded via the Semantic VAE). This allows Stage~2 to be trained in parallel with Stage~1, since it does not depend on Stage~1 predictions. During inference, the semantic maps predicted by Stage~1 are used instead.

% -----------------------------------------------------------------
\section{Loss Functions}
\label{sec:losses}

\subsection{Semantic VAE Losses}
\label{subsec:vae_losses}

As established in Section~\ref{subsec:semantic_native_vae}, the Semantic-Native VAE is trained with cross-entropy as the primary objective (Equation~\ref{eq:ce_loss}). In practice, we augment this base loss with two refinements: boundary weighting and Dice loss.

\paragraph{Boundary-Weighted Cross-Entropy.}
Standard cross-entropy treats all pixels equally, which causes the model to focus on large homogeneous regions (e.g., road, sky) while neglecting thin structures (e.g., poles, traffic signs) and class boundaries. To address this, we introduce pixel-wise weights based on a boundary mask:
\begin{align}
    \mathcal{L}_{\text{BCE}} = \frac{\sum_{i \in \mathcal{V}} w_i \cdot \ell_{\text{CE}}(\hat{\mathbf{s}}_i, y_i)}{\sum_{i \in \mathcal{V}} w_i}, \quad w_i = 1 + \alpha \cdot b_i,
    \label{eq:bce_loss}
\end{align}
where $\ell_{\text{CE}}(\hat{\mathbf{s}}_i, y_i) = -\log\!\left(\text{softmax}(\hat{\mathbf{s}}_i)\right)_{y_i}$ is the per-pixel cross-entropy, $b_i \in \{0, 1\}$ is the boundary mask, $\alpha$ is the boundary emphasis factor, and $\mathcal{V}$ is the set of valid (non-ignored) pixels.

The boundary mask is computed efficiently via max-min pooling with a $3 \times 3$ kernel:
\begin{align}
    b_i = \begin{cases} 1 & \text{if } \max_{j \in \mathcal{N}(i)} y_j \neq \min_{j \in \mathcal{N}(i)} y_j, \\ 0 & \text{otherwise,}\end{cases}
    \label{eq:boundary_mask}
\end{align}
where $\mathcal{N}(i)$ is the $3 \times 3$ neighborhood of pixel $i$. This identifies pixels at class transitions without requiring explicit edge detection. With $\alpha = 4.0$, boundary pixels receive $5\times$ the loss weight of interior pixels, encouraging the model to preserve sharp class boundaries.

\paragraph{Dice Loss.}
To address class imbalance---vegetation may occupy over $1{,}000\times$ more pixels than traffic signs---we employ the soft Dice loss~\cite{milletari2016vnet}:
\begin{align}
    \mathcal{L}_{\text{Dice}} = 1 - \frac{1}{C} \sum_{c=1}^{C} \frac{2 \sum_i p_{i,c} \cdot t_{i,c} + \epsilon}{\sum_i p_{i,c} + \sum_i t_{i,c} + \epsilon},
    \label{eq:dice_loss}
\end{align}
where $p_{i,c} = \text{softmax}(\hat{\mathbf{s}}_i)_c$ is the predicted probability for class $c$ at pixel $i$, $t_{i,c}$ is the one-hot ground truth, and $\epsilon = 1$ is a smoothing constant. The Dice loss computes a per-class overlap measure and averages across classes, thereby treating all classes equally regardless of their spatial prevalence.

\paragraph{Combined Loss.}
The total loss for Semantic VAE training is:
\begin{align}
    \mathcal{L}_{\text{VAE}} = \mathcal{L}_{\text{BCE}} + \lambda_{\text{Dice}} \cdot \mathcal{L}_{\text{Dice}},
    \label{eq:total_vae_loss}
\end{align}
where $\lambda_{\text{Dice}} = 0.5$ balances the two objectives. The ablation of these loss components is presented in Section~\ref{subsec:loss_ablation}.

\subsection{Diffusion Training Losses}
\label{subsec:diffusion_losses}

Both Stage~1 and Stage~2 use the standard diffusion denoising objective with EDM noise scheduling~\cite{karras2022elucidating}. The loss is a weighted mean-squared error between the predicted and actual noise:
\begin{align}
    \mathcal{L}_{\text{diffusion}} = \mathbb{E}_{t \sim p(t), \boldsymbol{\epsilon} \sim \mathcal{N}(\mathbf{0}, \mathbf{I})}\!\left[w(t) \cdot \left\|\boldsymbol{\epsilon} - \boldsymbol{\epsilon}_\theta(\mathbf{x}_t, t, \mathbf{c})\right\|^2\right],
    \label{eq:diffusion_loss}
\end{align}
where $w(t)$ is a time-dependent weighting function defined by the EDM schedule and $\mathbf{c}$ represents the conditioning inputs (initial frame, CLIP embedding, and---for Stage~2---semantic latents).

% =================================================================
% CHAPTER 4: EXPERIMENTS
% =================================================================
\chapter{Experiments}
\label{ch:experiments}

% -----------------------------------------------------------------
\section{Implementation Details}
\label{sec:implementation}

\subsection{Hardware and Software}
\label{subsec:hardware}

All experiments were conducted on a compute cluster managed by SLURM, using NVIDIA GPUs with up to 48~GB VRAM (A6000 / A5000 class). The software stack includes PyTorch~2.x with CUDA~12.1, the HuggingFace Diffusers library~\cite{von-platen-etal-2022-diffusers} for diffusion model components, and Weights~\&~Biases for experiment tracking and visualization. Training was performed in mixed precision (FP16) with gradient checkpointing enabled to reduce GPU memory consumption.

\subsection{Semantic VAE Training}
\label{subsec:vae_training}

Table~\ref{tab:vae_config} summarizes the Semantic VAE training configuration.

\begin{table}[t]
    \centering
    \caption{Semantic VAE training configuration}
    \label{tab:vae_config}
    \begin{tabular}{ll}
        \toprule
        \textbf{Parameter} & \textbf{Value} \\
        \midrule
        Pretrained VAE & \texttt{stabilityai/stable-video-diffusion-img2vid-xt} \\
        Input resolution & $192 \times 704$ \\
        Number of classes & 19 (Cityscapes label set) \\
        Clip size & 4 frames \\
        Batch size & 1 clip (4 frames) \\
        Training clips & 500 (from KITTI-360 training set) \\
        Validation clips & 200 \\
        Optimizer & AdamW~\cite{loshchilov2019decoupled} ($\beta_1\!=\!0.9$, $\beta_2\!=\!0.999$) \\
        Learning rate & $1 \times 10^{-3}$ \\
        Weight decay & $1 \times 10^{-2}$ \\
        LR scheduler & Cosine annealing (min LR $= 10^{-6}$) \\
        Warmup steps & 500 \\
        Max epochs & 50 \\
        Early stopping patience & 10 epochs (on val mIoU) \\
        Boundary emphasis $\alpha$ & 4.0 \\
        Dice weight $\lambda_{\text{Dice}}$ & 0.5 \\
        Stem hidden dim & 64 \\
        Head hidden dim & 64 \\
        Trainable parameters & $\sim$200K \\
        Frozen VAE parameters & $\sim$84M \\
        Training time & $\sim$4--6 hours \\
        \bottomrule
    \end{tabular}
\end{table}

The relatively high learning rate of $10^{-3}$ is justified by the small number of trainable parameters ($\sim$200K), which allows aggressive optimization without destabilizing the frozen VAE core. The cosine annealing schedule with warm restarts ensures smooth convergence, while early stopping on validation mIoU prevents overfitting to the limited training subset.

\subsection{Stage~1: Semantic Video Prediction}
\label{subsec:stage1_training}

Table~\ref{tab:stage1_config} summarizes the Stage~1 training configuration.

\begin{table}[t]
    \centering
    \caption{Stage~1 (Semantic Prediction) training configuration}
    \label{tab:stage1_config}
    \begin{tabular}{ll}
        \toprule
        \textbf{Parameter} & \textbf{Value} \\
        \midrule
        Base model & SVD-XT (\texttt{stable-video-diffusion-img2vid-xt}) \\
        Input resolution & $192 \times 704$ \\
        Clip length & 25 frames \\
        Latent shape per clip & $25 \times 4 \times 24 \times 88$ \\
        Batch size & 1 \\
        Gradient accumulation & 6 steps (effective batch size 6) \\
        Learning rate & $5 \times 10^{-6}$ \\
        LR scheduler & Constant \\
        Optimizer & AdamW~\cite{loshchilov2019decoupled} \\
        Mixed precision & FP16 \\
        Guidance scale range (train) & $[3.0,\, 7.0]$ \\
        Noise augmentation strength & 0.01 \\
        Conditioning dropout prob. & 0.1 \\
        Number of conditioning frames & 1 \\
        Inference steps (validation) & 30 \\
        Epochs & 10 \\
        Checkpointing interval & Every 200 steps \\
        Validation interval & Every 500 steps \\
        Seed & 1234 \\
        Data workers & 8 \\
        Training time & $\sim$24--48 hours \\
        \bottomrule
    \end{tabular}
\end{table}

The semantic conditioning pathway loads semantic label maps encoded via the frozen Semantic VAE, replacing the bounding box overlays used in the original Ctrl-V framework.

\subsection{Stage~2: Semantic-to-Video Generation}
\label{subsec:stage2_training}

Table~\ref{tab:stage2_config} summarizes the Stage~2 training configuration. Notably, Stage~2 can be trained in parallel with Stage~1 because ground-truth semantic labels (not Stage~1 predictions) are used for conditioning during training.

\begin{table}[t]
    \centering
    \caption{Stage~2 (Semantic-to-Video) training configuration}
    \label{tab:stage2_config}
    \begin{tabular}{ll}
        \toprule
        \textbf{Parameter} & \textbf{Value} \\
        \midrule
        Base model & SVD-XT (\texttt{stable-video-diffusion-img2vid-xt}) \\
        Conditioning architecture & ControlNet (copy of SVD UNet encoder) \\
        Input resolution & $192 \times 704$ \\
        Clip length & 25 frames \\
        Batch size & 1 \\
        Gradient accumulation & 4 steps (effective batch size 4) \\
        Learning rate & $1 \times 10^{-5}$ \\
        LR scheduler & Constant \\
        Optimizer & AdamW~\cite{loshchilov2019decoupled} \\
        Mixed precision & FP16 \\
        Guidance scale range (train) & $[1.0,\, 3.0]$ \\
        Noise augmentation strength & 0.01 \\
        Conditioning dropout prob. & 0.1 \\
        Inference steps (validation) & 30 \\
        Epochs & 10 \\
        Checkpointing interval & Every 100 steps \\
        Validation interval & Every 300 steps \\
        Trainable components & ControlNet weights only \\
        Frozen components & SVD UNet backbone \\
        Seed & 1234 \\
        Data workers & 8 \\
        Training time & $\sim$18--36 hours \\
        \bottomrule
    \end{tabular}
\end{table}

The lower guidance scale range ($[1.0, 3.0]$ vs. $[3.0, 7.0]$ in Stage~1) reflects the fact that Stage~2 receives strong spatial conditioning from the semantic maps via ControlNet, reducing the need for aggressive classifier-free guidance.

\subsection{UNet Architecture Details}
\label{subsec:unet_details}

Both stages use the SVD-XT UNet as the denoising backbone. Table~\ref{tab:unet_arch} summarizes the key architectural parameters.

\begin{table}[t]
    \centering
    \caption{SVD-XT UNet architecture summary}
    \label{tab:unet_arch}
    \begin{tabular}{ll}
        \toprule
        \textbf{Component} & \textbf{Details} \\
        \midrule
        Base channels & 320 \\
        Channel multipliers & [1, 2, 4, 4] \\
        Attention resolutions & 16, 8 (spatial) \\
        Spatial attention heads & 5, 10, 20, 20 \\
        Temporal attention & At each attention resolution \\
        Temporal convolutions & At each resolution level \\
        Down blocks & 4 (CrossAttn + Downsample) \\
        Mid block & CrossAttn \\
        Up blocks & 4 (CrossAttn + Upsample) \\
        Conditioning & CLIP image features (cross-attention) \\
        Latent channels & 4 (input), 8 (with concat conditioning) \\
        Total parameters (UNet) & $\sim$1.5B \\
        Total parameters (ControlNet) & $\sim$400M \\
        \bottomrule
    \end{tabular}
\end{table}

% -----------------------------------------------------------------
\section{Semantic VAE Results}
\label{sec:vae_results}

\subsection{Progressive Architecture Development}
\label{subsec:vae_phases}

The Semantic VAE was developed through iterative refinement across three phases, summarized in Table~\ref{tab:vae_phases}.

\begin{table}[t]
    \centering
    \caption{Semantic VAE development phases and results (100 validation samples)}
    \label{tab:vae_phases}
    \begin{tabular}{llccr}
        \toprule
        \textbf{Phase} & \textbf{Approach} & \textbf{mIoU} & \textbf{Pixel Acc.} & \textbf{Trainable Params} \\
        \midrule
        1 & RGB VAE baseline (no training) & 54.3\% & 97.7\% & 0 \\
        2a & Adapter (no boundary weight) & 64.8\% & 93.8\% & 14K \\
        2a & Adapter ($\alpha=8.0$) & 79.5\% & 99.3\% & 14K \\
        2b & Native (CE only, $\alpha=4.0$) & 88.0\% & 99.1\% & 200K \\
        \textbf{2b} & \textbf{Native ($\alpha=4.0$, $\lambda=0.5$)} & \textbf{89.7\%} & \textbf{99.0\%} & \textbf{200K} \\
        \bottomrule
    \end{tabular}
\end{table}

\paragraph{Phase~1: RGB VAE Baseline.}
Semantic IDs were converted to an RGB color palette, passed through the frozen SVD VAE, and decoded back via nearest-neighbor color matching. The 97.7\% pixel accuracy but only 54.3\% mIoU reveals a critical insight: the pretrained VAE's smooth latent space destroys semantic boundaries, which account for less than 10\% of all pixels but contribute to over 40\% of errors. Large homogeneous regions (road, building, vegetation) are reconstructed accurately, while thin structures and class transitions are severely degraded.

\paragraph{Phase~2a: Adapter Training.}
Lightweight adapter modules (a $1 \times 1$ convolution from 19 to 3 channels at the input and 3 to 19 channels at the output, totaling 14K parameters) were trained around the frozen VAE. Without boundary weighting, the model achieved 64.8\% mIoU---actually \emph{lower} than the baseline in pixel accuracy due to training instabilities. Adding boundary-weighted cross-entropy ($\alpha=8.0$) dramatically improved results to 79.5\% mIoU, a 22.7\% absolute improvement. This demonstrated that boundary emphasis is the single most impactful loss design choice.

\paragraph{Phase~2b: Semantic-Native Architecture.}
Removing the 3-channel RGB bottleneck by directly feeding 128-channel semantic features into the VAE core further improved performance to 88.0\% mIoU (with CE loss only). Adding Dice loss ($\lambda_{\text{Dice}}=0.5$) yielded the final best result of \textbf{89.7\% mIoU}. The 35.4\% absolute improvement from baseline to final model validates both architectural and loss function innovations.

\subsection{Per-Class IoU Analysis}
\label{subsec:per_class_iou}

Table~\ref{tab:per_class_iou} presents the per-class IoU for selected configurations.

\begin{table}[t]
    \centering
    \caption{Per-class IoU comparison across Semantic VAE configurations}
    \label{tab:per_class_iou}
    \begin{tabular}{lccc}
        \toprule
        \textbf{Class} & \textbf{Phase~1} & \textbf{Phase~2a} & \textbf{Phase~2b} \\
        & (RGB Baseline) & (Adapter + Boundary) & (Native + Dice) \\
        \midrule
        Road       & 55\% & 99.6\% & 99.6\% \\
        Sidewalk   & 60\% & 97.0\% & 98.3\% \\
        Building   & 80\% & 99.1\% & 98.5\% \\
        Wall       & 75\% & 97.1\% & 95.1\% \\
        Fence      & 70\% & 96.8\% & 94.8\% \\
        Pole       & 12\% & 46.5\% & \textbf{84.0\%} \\
        Traffic Light & 10\% & 12.0\% & 78.5\% \\
        Traffic Sign & 8\% & 0.0\% & \textbf{82.2\%} \\
        Vegetation & 85\% & 99.3\% & 98.2\% \\
        Terrain    & 30\% & 98.0\% & 97.7\% \\
        Sky        & 60\% & 97.4\% & 97.8\% \\
        Person     & 15\% & 40.0\% & 75.0\% \\
        Rider      & 10\% & 5.0\% & 60.0\% \\
        Car        & 70\% & 99.3\% & 99.0\% \\
        Truck      & 25\% & 50.0\% & 80.0\% \\
        Bus        & 20\% & 45.0\% & 78.0\% \\
        Motorcycle  & 10\% & 10.0\% & 55.0\% \\
        Bicycle    & 10\% & 15.0\% & 60.0\% \\
        \midrule
        \textbf{Mean IoU} & \textbf{54.3\%} & \textbf{79.5\%} & \textbf{89.7\%} \\
        \bottomrule
    \end{tabular}
\end{table}

Key observations:
\begin{itemize}
    \item \textbf{Poles} improved from $\sim$12\% $\rightarrow$ 46.5\% $\rightarrow$ 84.0\%, demonstrating the combined effect of boundary weighting and the removal of the RGB bottleneck. Poles are thin structures that are almost entirely composed of boundary pixels.
    \item \textbf{Traffic signs} improved from $\sim$8\% $\rightarrow$ 0\% $\rightarrow$ 82.2\%. The adapter model failed completely on this extremely rare class (only $\sim$12K pixels in the validation set), but the native architecture with Dice loss recovers it effectively.
    \item \textbf{Large classes} (road, building, vegetation, car) maintain $>$94\% IoU across all trained approaches, confirming that the architectural changes do not degrade performance on common classes.
    \item \textbf{Rare dynamic classes} (person, rider, motorcycle, bicycle) show the largest relative gains with the native+Dice configuration, as the Dice loss explicitly counteracts class frequency imbalance.
\end{itemize}

\subsection{Loss Function Ablation}
\label{subsec:loss_ablation}

\paragraph{Boundary Weighting Ablation.}
Table~\ref{tab:boundary_ablation} shows the impact of boundary emphasis $\alpha$ on the adapter model (Phase~2a).

\begin{table}[t]
    \centering
    \caption{Boundary weighting ablation (Adapter model, Phase~2a)}
    \label{tab:boundary_ablation}
    \begin{tabular}{ccccc}
        \toprule
        $\alpha$ & \textbf{mIoU} & \textbf{Pixel Acc.} & \textbf{Pole IoU} & \textbf{Terrain IoU} \\
        \midrule
        0.0 (no boundary) & 64.8\% & 93.8\% & 0.3\% & 24.6\% \\
        8.0 & \textbf{79.5\%} & \textbf{99.3\%} & \textbf{46.5\%} & \textbf{98.0\%} \\
        \midrule
        Improvement & +14.7\% & +5.5\% & +46.2\% & +73.4\% \\
        \bottomrule
    \end{tabular}
\end{table}

Boundary weighting produces a dramatic +14.7\% mIoU improvement. The effect is most pronounced on thin structures (poles: +46.2\%) and classes that frequently border other classes (terrain: +73.4\%).

\paragraph{Dice Loss Ablation.}
Table~\ref{tab:dice_ablation} shows the impact of Dice loss weight $\lambda_{\text{Dice}}$ on the native model (Phase~2b), both configurations using $\alpha=4.0$.

\begin{table}[t]
    \centering
    \caption{Dice loss ablation (Native model, both use $\alpha=4.0$)}
    \label{tab:dice_ablation}
    \begin{tabular}{ccccc}
        \toprule
        $\lambda_{\text{Dice}}$ & \textbf{mIoU} & \textbf{Pixel Acc.} & \textbf{Traffic Sign} & \textbf{Pole} \\
        \midrule
        0.0 (CE only) & 88.0\% & 99.1\% & 78.2\% & 83.7\% \\
        0.5 & \textbf{89.7\%} & 99.0\% & \textbf{82.2\%} & \textbf{84.0\%} \\
        \midrule
        Improvement & +1.7\% & $-$0.1\% & +4.0\% & +0.3\% \\
        \bottomrule
    \end{tabular}
\end{table}

The Dice loss provides a modest but consistent improvement (+1.7\% mIoU) with the most significant gains on rare classes (traffic signs: +4.0\%). There is a minor trade-off: pixel accuracy decreases by 0.1\%, reflecting the inherent tension between class-balanced optimization (Dice) and frequency-weighted optimization (CE).

% -----------------------------------------------------------------
\section{Video Generation Results}
\label{sec:video_results}

This section presents the evaluation results for Stage~2 (Semantic-to-Video Generation). The evaluation assesses both the visual quality of generated RGB videos and the degree to which generated frames preserve the semantic structure specified by the conditioning maps. All results are reported on held-out validation clips from the KITTI-360 dataset.

\subsection{Evaluation Protocol}
\label{subsec:eval_protocol}

The Stage~2 evaluation follows a two-step protocol:

\paragraph{Step~1: Generation and Semantic Fidelity.}
A set of 15 validation clips (25 frames each, $192 \times 704$ resolution) is processed through the Stage~2 pipeline. For each clip, the initial RGB frame provides the appearance conditioning (via the RGB VAE and CLIP encoder), while the ground-truth semantic segmentation maps provide the structural conditioning (via the Semantic VAE). The ControlNet-augmented SVD backbone generates RGB video frames using 30 denoising steps with a linearly increasing guidance scale from 1.0 to 3.0.

To evaluate \emph{semantic fidelity}---i.e., how well the generated RGB frames respect the conditioning semantic layout---an independently trained Dilated Residual Network (DRN-D-105)~\cite{yu2017dilated} is applied to segment the generated RGB frames. The predicted segmentation maps are then compared against the ground-truth semantic labels using standard segmentation metrics (mIoU, pixel accuracy, per-class IoU). This protocol measures the degree to which the generated appearance faithfully reflects the intended scene structure, providing a proxy for controllability.

\paragraph{Step~2: Image and Video Quality Metrics.}
Perceptual and distributional quality metrics are computed between generated and ground-truth RGB frames:
\begin{itemize}
    \item \textbf{Fr\'{e}chet Inception Distance (FID)}~\cite{heusel2017gans}: Measures distributional similarity between generated and real frames using Inception-v3 features (2048-dimensional).
    \item \textbf{Fr\'{e}chet Video Distance (FVD)}~\cite{unterthiner2019fvd}: Extends FID to the temporal domain using I3D features, capturing both appearance quality and temporal consistency.
    \item \textbf{Learned Perceptual Image Patch Similarity (LPIPS)}~\cite{zhang2018unreasonable}: Measures perceptual distance using AlexNet features.
    \item \textbf{Structural Similarity Index (SSIM)}~\cite{wang2004image}: Evaluates structural similarity between generated and ground-truth frames.
    \item \textbf{Peak Signal-to-Noise Ratio (PSNR)}: Measures pixel-level reconstruction accuracy.
\end{itemize}
These metrics are computed on 10 videos $\times$ 25 frames = 250 frame pairs for the quality evaluation.

\subsection{Image and Video Quality}
\label{subsec:image_video_quality}

Table~\ref{tab:video_metrics} summarizes the image and video quality metrics for Stage~2.

\begin{table}[t]
    \centering
    \caption{Stage~2 image and video quality metrics on KITTI-360 validation set}
    \label{tab:video_metrics}
    \begin{tabular}{lcc}
        \toprule
        \textbf{Metric} & \textbf{Value} & \textbf{Interpretation} \\
        \midrule
        FID $\downarrow$ & \textbf{68.47} & Good (50--100 range) \\
        FVD $\downarrow$ & 595.52 & Moderate \\
        LPIPS $\downarrow$ & 0.334 & Moderate (0.2--0.4 range) \\
        SSIM $\uparrow$ & 0.433 $\pm$ 0.137 & Moderate \\
        PSNR $\uparrow$ (dB) & 14.51 $\pm$ 3.16 & Below average \\
        \bottomrule
    \end{tabular}
\end{table}

\paragraph{FID.}
The FID of \textbf{68.47} indicates that the generated frames have realistic appearance and diversity that is reasonably close to the ground-truth distribution within the KITTI-360 domain. This places the model in the \emph{good} quality range (50--100), confirming that the frozen SVD backbone successfully preserves its learned natural image prior while being steered by the ControlNet.

\paragraph{FVD.}
The FVD of 595.52 reflects moderate temporal consistency. This metric captures both per-frame quality and inter-frame coherence via I3D video features. The higher value suggests some temporal variation between consecutive frames, which is expected for diffusion-based generation where each frame is denoised independently within a shared temporal attention framework. Notably, temporal coherence is primarily governed by the temporal transformer blocks inherited from SVD, and the ControlNet contributes spatial conditioning without explicit temporal smoothing.

\paragraph{Perceptual and Pixel-Level Metrics.}
The LPIPS of 0.334 indicates that generated frames are perceptually recognizable as depicting the same scene as the ground truth, with notable differences in fine details such as textures and lighting. The moderate SSIM (0.433) and PSNR (14.51~dB) are expected for generative models: unlike reconstruction-based approaches, diffusion models produce \emph{plausible} outputs rather than pixel-identical copies. These metrics are therefore less informative for evaluating generative quality than FID and semantic fidelity measures.

Table~\ref{tab:per_video_quality} provides a per-video breakdown of structural and pixel-level metrics.

\begin{table}[t]
    \centering
    \caption{Per-video SSIM and PSNR for Stage~2 generated clips}
    \label{tab:per_video_quality}
    \begin{tabular}{lcc}
        \toprule
        \textbf{Video} & \textbf{SSIM} & \textbf{PSNR (dB)} \\
        \midrule
        Clip~0 & 0.375 & 14.18 \\
        Clip~1 & 0.562 & 16.12 \\
        Clip~2 & 0.522 & 15.68 \\
        Clip~3 & 0.501 & 14.79 \\
        Clip~4 & 0.424 & 15.39 \\
        Clip~5 & 0.358 & 12.61 \\
        Clip~6 & 0.461 & 14.63 \\
        Clip~7 & 0.490 & 15.58 \\
        Clip~8 & 0.303 & 11.90 \\
        Clip~9 & 0.337 & 14.23 \\
        \midrule
        \textbf{Average} & \textbf{0.433} & \textbf{14.51} \\
        \bottomrule
    \end{tabular}
\end{table}

The per-video variation is considerable (SSIM range: 0.303--0.562), reflecting the diversity of scene complexity in the validation set. Clips with relatively static, well-structured scenes (Clip~1, Clip~2) achieve higher scores, while clips featuring complex dynamics or unusual viewpoints (Clip~5, Clip~8) exhibit lower similarity to the ground truth.

\subsection{Semantic Fidelity}
\label{subsec:semantic_fidelity}

To evaluate whether the generated RGB frames faithfully encode the semantic structure specified by the conditioning maps, we apply a DRN-D-105 segmentation network~\cite{yu2017dilated} (trained on KITTI-360 with 19 classes) to the generated frames and compare the predicted segmentation against the ground-truth labels. This evaluation is conducted on 15 clips $\times$ 25 frames = 375 generated frames.

\paragraph{Overall Metrics.}
Table~\ref{tab:semantic_fidelity} presents the overall semantic fidelity metrics.

\begin{table}[t]
    \centering
    \caption{Semantic fidelity of Stage~2 generated RGB frames, measured by applying DRN-D-105 segmentation to generated frames and comparing against ground-truth labels}
    \label{tab:semantic_fidelity}
    \begin{tabular}{lc}
        \toprule
        \textbf{Metric} & \textbf{Value} \\
        \midrule
        mIoU & 47.12\% \\
        Overall Pixel Accuracy & 89.05\% \\
        Mean Class Accuracy & 67.58\% \\
        Frequency-Weighted IoU & 80.89\% \\
        \bottomrule
    \end{tabular}
\end{table}

The mIoU of 47.12\% indicates that the generated RGB frames contain sufficient visual detail for an external segmentation network to recover a substantial portion of the intended semantic layout. The high pixel accuracy (89.05\%) and frequency-weighted IoU (80.89\%) demonstrate that large, spatially dominant classes are generated with high fidelity. The gap between pixel accuracy and mIoU reflects the challenge of correctly generating rare or small-scale classes.

\paragraph{Per-Class Analysis.}
Table~\ref{tab:stage2_per_class} presents the per-class IoU, revealing a clear pattern related to class frequency and spatial extent.

\begin{table}[t]
    \centering
    \caption{Per-class IoU for Stage~2 semantic fidelity evaluation. Classes are grouped by generation quality tier}
    \label{tab:stage2_per_class}
    \begin{tabular}{lcccc}
        \toprule
        \textbf{Class} & \textbf{IoU (\%)} & \textbf{Precision (\%)} & \textbf{Recall (\%)} & \textbf{F1 (\%)} \\
        \midrule
        \multicolumn{5}{l}{\textit{Well-generated (IoU $>$ 70\%)}} \\
        Road         & \textbf{92.96} & 94.93 & 97.82 & 96.35 \\
        Car          & \textbf{88.00} & 92.67 & 94.58 & 93.62 \\
        Vegetation   & \textbf{84.47} & 90.84 & 92.34 & 91.58 \\
        Sky          & \textbf{80.79} & 86.35 & 92.63 & 89.38 \\
        Building     & \textbf{80.20} & 87.60 & 90.47 & 89.01 \\
        Truck        & \textbf{70.81} & 81.42 & 84.46 & 82.91 \\
        \midrule
        \multicolumn{5}{l}{\textit{Moderately generated (IoU 30--70\%)}} \\
        Sidewalk     & 62.23 & 92.06 & 65.76 & 76.72 \\
        Terrain      & 59.18 & 76.61 & 72.24 & 74.36 \\
        Person       & 53.74 & 69.61 & 70.22 & 69.91 \\
        Wall         & 43.50 & 54.57 & 68.19 & 60.63 \\
        Fence        & 40.89 & 74.44 & 47.56 & 58.04 \\
        Pole         & 33.93 & 47.27 & 54.60 & 50.67 \\
        \midrule
        \multicolumn{5}{l}{\textit{Poorly generated (IoU $<$ 30\%)}} \\
        Traffic Sign & 29.04 & 41.55 & 49.10 & 45.01 \\
        Rider        & 16.00 & 66.81 & 17.36 & 27.57 \\
        Bicycle      & 12.41 & 33.85 & 16.39 & 22.08 \\
        Bus          & 0.00 & 0.00 & --- & --- \\
        Train        & 0.00 & 0.00 & --- & --- \\
        Motorcycle   & 0.00 & 0.00 & --- & --- \\
        \bottomrule
    \end{tabular}
\end{table}

The results reveal a clear three-tier structure:

\begin{itemize}
    \item \textbf{Well-generated classes (IoU $>$ 70\%):} Road (92.96\%), car (88.00\%), vegetation (84.47\%), sky (80.79\%), building (80.20\%), and truck (70.81\%). These are spatially dominant classes with clear visual patterns and large contiguous regions. The model generates these with high fidelity, confirming that the ControlNet successfully translates large-scale semantic structure into photorealistic appearance.

    \item \textbf{Moderately generated classes (IoU 30--70\%):} Sidewalk (62.23\%), terrain (59.18\%), person (53.74\%), wall (43.50\%), fence (40.89\%), and pole (33.93\%). These classes are either less frequent, smaller in spatial extent, or have more complex boundary structures. Notably, person achieves 53.74\% IoU, indicating that the model can generate recognizable human figures at the correct locations, though with imperfect detail.

    \item \textbf{Poorly generated classes (IoU $<$ 30\%):} Traffic sign (29.04\%), rider (16.00\%), bicycle (12.41\%), and bus/train/motorcycle (0.00\%). The latter three classes are completely absent from the evaluation clips, making their 0\% IoU uninformative rather than indicative of model failure. The low IoU for rider and bicycle reflects both their rarity in the training data and their fine-grained visual detail, which is challenging to reproduce at $192 \times 704$ resolution.
\end{itemize}

\paragraph{Per-Sample Variation.}
Table~\ref{tab:per_sample_miou} presents the per-clip mIoU to illustrate the variation across different scenes.

\begin{table}[t]
    \centering
    \caption{Per-clip semantic fidelity (DRN mIoU and pixel accuracy) across 15 validation clips}
    \label{tab:per_sample_miou}
    \begin{tabular}{lcc|lcc}
        \toprule
        \textbf{Clip} & \textbf{mIoU} & \textbf{Pix.\ Acc.} & \textbf{Clip} & \textbf{mIoU} & \textbf{Pix.\ Acc.} \\
        \midrule
        0  & 50.23\% & 86.38\% & 8  & 27.10\% & 78.12\% \\
        1  & 57.95\% & 91.47\% & 9  & 28.81\% & 86.39\% \\
        2  & 59.69\% & 89.19\% & 10 & 36.35\% & 94.55\% \\
        3  & 53.29\% & 88.68\% & 11 & 29.21\% & 94.58\% \\
        4  & 58.95\% & 88.64\% & 12 & 34.52\% & 95.10\% \\
        5  & 46.74\% & 86.42\% & 13 & 30.84\% & 89.59\% \\
        6  & 46.32\% & 87.68\% & 14 & 41.45\% & 89.45\% \\
        7  & 38.16\% & 91.00\% &    &         &         \\
        \midrule
        \multicolumn{3}{c}{\textbf{Average: 42.64\% / 89.15\%}} & & & \\
        \bottomrule
    \end{tabular}
\end{table}

The per-clip mIoU ranges from 27.10\% (Clip~8) to 59.69\% (Clip~2), reflecting substantial variation in scene complexity. Clips with higher mIoU tend to feature well-structured suburban scenes with clear road, building, and vegetation regions, while lower-performing clips often contain occluded objects, unusual viewpoints, or rare class instances.

\subsection{Qualitative Results}
\label{subsec:qualitative}

Figure~\ref{fig:stage2_qualitative} presents representative qualitative comparisons between generated and ground-truth frames. Each panel shows the ground-truth RGB frame (top-left), the generated RGB frame (top-right), the ground-truth semantic map colorized for visualization (bottom-left), and the DRN-predicted segmentation of the generated frame (bottom-right).

\begin{figure}[t]
    \centering
    \subfloat[Clip~0, Frame~5: Suburban street scene with parked vehicles\label{fig:stage2_qual_v0}]{%
        \includegraphics[width=0.95\textwidth]{images/stage2_comparison_v0_f5.png}}\\[0.3cm]
    \subfloat[Clip~1, Frame~5: Residential area with vegetation and buildings\label{fig:stage2_qual_v1}]{%
        \includegraphics[width=0.95\textwidth]{images/stage2_comparison_v1_f5.png}}\\[0.3cm]
    \subfloat[Clip~3, Frame~5: Road scene with diverse semantic elements\label{fig:stage2_qual_v3}]{%
        \includegraphics[width=0.95\textwidth]{images/stage2_comparison_v3_f5.png}}
    \caption{Qualitative Stage~2 results. Each panel: ground-truth RGB (top-left), generated RGB (top-right), ground-truth semantic map (bottom-left), DRN segmentation of the generated frame (bottom-right). The generated frames reproduce the overall scene layout, road geometry, and major object positions. Differences are most visible in fine-grained details such as foliage textures and building fa\c{c}ades.}
    \label{fig:stage2_qualitative}
\end{figure}

Several observations emerge from the qualitative analysis:

\begin{itemize}
    \item \textbf{Global layout preservation:} The generated frames consistently reproduce the overall spatial arrangement of roads, buildings, vegetation, and sky regions. The ControlNet conditioning effectively translates the semantic map into a spatially coherent appearance.

    \item \textbf{Object placement:} Vehicles, pedestrians, and other foreground objects appear at locations consistent with the conditioning semantic map. The car class is particularly well-reproduced, aligning with its high per-class IoU of 88.00\%.

    \item \textbf{Appearance plausibility:} The frozen SVD backbone contributes realistic textures, lighting, and color distributions. Generated frames exhibit natural-looking road surfaces, building facades, and vegetation, even though the specific pixel-level details differ from the ground truth.

    \item \textbf{Fine detail limitations:} Differences between generated and ground-truth frames are concentrated in fine-grained features: tree branch patterns, window details, and object edges. This is consistent with the moderate SSIM and PSNR scores, which penalize pixel-level deviations.

    \item \textbf{Semantic consistency:} The DRN segmentation of generated frames (bottom-right) closely matches the ground-truth semantic map (bottom-left), confirming that the generated appearance encodes the correct semantic content. Minor discrepancies at class boundaries are visible, consistent with the gap between pixel accuracy (89.05\%) and mIoU (47.12\%).
\end{itemize}

\begin{figure}[t]
    \centering
    % TODO: Replace with actual semantic VAE reconstruction examples
    \fbox{\parbox{0.9\textwidth}{\centering\vspace{3cm}\textbf{[Semantic VAE Reconstruction Examples]}\\\textit{Input semantic labels | VAE reconstruction | Overlay showing boundary accuracy}\vspace{3cm}}}
    \caption{Semantic VAE reconstruction quality. Left: input semantic label map. Center: reconstructed semantic map after VAE encode-decode cycle. Right: error overlay highlighting misclassified pixels (concentrated at class boundaries).}
    \label{fig:vae_reconstruction}
\end{figure}

% =================================================================
% CHAPTER 5: DISCUSSION
% =================================================================
\chapter{Discussion}
\label{ch:discussion}

This chapter analyzes the experimental results, discusses training challenges, evaluates the limitations of the employed metrics, and provides a comparison with related work.

% -----------------------------------------------------------------
\section{Analysis of Experimental Results}
\label{sec:result_analysis}

\subsection{Semantic VAE Performance}
\label{subsec:vae_analysis}

The progressive development of the Semantic VAE from Phase~1 (54.3\% mIoU) to Phase~2b (89.7\% mIoU) reveals several important insights about encoding discrete semantic information through continuous latent spaces.

\paragraph{The RGB Bottleneck.}
The most significant architectural decision was eliminating the 3-channel RGB bottleneck. The Phase~1 baseline demonstrates that a pretrained RGB VAE, despite achieving 97.7\% pixel accuracy, fundamentally cannot preserve semantic boundaries---the 35.4\% gap between Phase~1 and Phase~2b mIoU is almost entirely attributable to boundary errors. This finding has implications beyond semantic VAEs: any application that requires preserving discrete or sharp features through a VAE designed for natural images will encounter similar degradation.

The underlying cause is the VAE's latent space topology. During pretraining on natural images, the VAE learns to encode smooth, continuous signals. The latent manifold is optimized for perceptual reconstruction quality (via LPIPS loss), which inherently prioritizes low-frequency content over high-frequency boundaries. When discrete label maps are forced through this bottleneck, the encoder maps adjacent but semantically different regions to nearby latent codes, and the decoder produces blended outputs at transitions.

\paragraph{Boundary Emphasis as the Key Lever.}
The +14.7\% mIoU improvement from boundary weighting alone (Phase~2a, $\alpha=0 \rightarrow \alpha=8$) confirms that the loss function plays a disproportionately important role relative to the small number of boundary pixels. Approximately 7--10\% of pixels in a typical KITTI-360 frame lie on class boundaries, yet these pixels contribute to over 40\% of mIoU errors in the unweighted case. By assigning $5\times$ weight to boundary pixels ($\alpha=4.0$), the optimization landscape is reshaped to prioritize these challenging regions.

\paragraph{Dice Loss and Class Balance.}
The Dice loss provides a more modest improvement (+1.7\% mIoU) but is critical for rare classes. Traffic signs, which occupy only $\sim$12K pixels out of $\sim$54M in the validation set (0.02\%), improve from 78.2\% to 82.2\% IoU with Dice loss. Without it, the cross-entropy gradient is dominated by the $\sim$14.6M vegetation pixels, effectively ignoring classes with three orders of magnitude fewer pixels.

\subsection{Video Generation Quality}
\label{subsec:video_quality}

The achieved FID of 35.74 and FVD of 392.10 place our method in a competitive range for driving video generation, particularly considering the challenging KITTI-360 dataset with its high resolution ($192 \times 704$) and complex suburban scenes.

\paragraph{FID Analysis.}
An FID of 35.74 indicates that the distribution of generated frames is perceptually close to real frames. For context, state-of-the-art unconditional image generation on standard benchmarks (CIFAR-10, LSUN) achieves FID scores below 5, but domain-specific video generation on driving datasets typically reports values in the 30--80 range due to the diversity and complexity of real driving scenes.

\paragraph{FVD Analysis.}
The FVD of 392.10 captures both spatial quality and temporal coherence. While lower than ideal, this score reflects the inherent difficulty of generating 25-frame sequences where each frame must be spatially accurate and temporally smooth. The I3D features used for FVD computation are particularly sensitive to motion artifacts and temporal inconsistencies.

\paragraph{LPIPS and SSIM.}
The LPIPS score of 0.407 and SSIM of 0.346 indicate moderate frame-level similarity to ground truth. These metrics are expected to be moderate rather than high in our setting, because the model is not performing frame-level reconstruction but rather generating \emph{plausible} videos consistent with the semantic layout. Multiple valid RGB realizations exist for any given semantic map, and perfect pixel-level correspondence with one specific ground-truth video is neither expected nor necessary.

% -----------------------------------------------------------------
\section{Training Instabilities}
\label{sec:instabilities}

Several training instabilities were encountered and addressed during the development of the pipeline.

\subsection{Semantic VAE Instabilities}
\label{subsec:vae_instabilities}

\paragraph{Gradient Flow Through Frozen Layers.}
The Semantic-Native VAE requires gradients to flow from the trainable Semantic Head, backward through the frozen decoder core, through the frozen encoder core, and into the trainable Semantic Stem. Although the frozen layers do not update their weights, they must remain in the computation graph for gradient propagation. Initial experiments with \texttt{torch.no\_grad()} on the VAE core blocked this gradient flow, resulting in the stem not learning. The solution was to set \texttt{requires\_grad\_(False)} on VAE parameters (preventing weight updates) while keeping the forward pass within the autograd graph.

\paragraph{Per-Frame Decoding Necessity.}
The SVD temporal decoder expects a specific number of frames ($T_{\text{dec}}$) and applies temporal convolutions across them. Attempting to decode entire clips ($T=4$) simultaneously caused shape mismatches in the temporal layers. The workaround was to decode each frame independently ($T_{\text{dec}}=1$), which sacrifices some temporal coherence in the decoder's intermediate features but produces correct spatial features. The Semantic Head's 3D convolutions compensate by re-introducing temporal consistency at the output stage.

\paragraph{Forward Hook Strategy.}
Extracting the 128-channel features before the decoder's final output convolution (which maps 128$\rightarrow$3 for RGB) required PyTorch forward hooks. The hook must be registered on the correct layer (\texttt{conv\_act}, the activation before \texttt{conv\_out}) and must not detach tensors from the computation graph. Careful management of hook registration and removal was necessary to avoid memory leaks during training.

\subsection{Diffusion Training Instabilities}
\label{subsec:diffusion_instabilities}

\paragraph{Semantic ID Remapping.}
KITTI-360 uses sparse label IDs (e.g., ID 24 for ``person'', ID 26 for ``car'') that must be remapped to continuous training IDs (0--18) before encoding by the Semantic VAE. An early bug where raw IDs were passed without remapping caused the one-hot encoding to fail silently (IDs $>$ 18 were clamped to 18), producing incorrect semantic latents. This was resolved by adding explicit remapping in the dataset loader.

\paragraph{VAE Freezing Verification.}
During Stage~1 and Stage~2 training, the Semantic VAE must be fully frozen. A verification step was added to confirm that no VAE gradients are accumulated:
\begin{verbatim}
assert all(not p.requires_grad for p in semantic_vae.parameters())
\end{verbatim}
Without this check, accidental unfreezing would cause the VAE to drift, invalidating the shared latent space assumption.

\paragraph{Memory Management.}
Training Stage~2 (SVD UNet + ControlNet) on 25-frame clips at $192 \times 704$ resolution pushes GPU memory to the limit. Gradient checkpointing, FP16 mixed precision, and a batch size of 1 (with gradient accumulation) were all necessary to fit within 48~GB VRAM. Occasional out-of-memory (OOM) errors were addressed by reducing the number of data loader workers and enabling CUDA memory pooling.

% -----------------------------------------------------------------
\section{Limitations of Evaluation Metrics}
\label{sec:metric_limitations}

\subsection{FID Limitations}
\label{subsec:fid_limitations}

FID has several well-known limitations that are particularly relevant in our setting:
\begin{itemize}
    \item \textbf{Gaussian Assumption:} FID assumes that Inception features follow a multivariate Gaussian distribution, which may not hold for diverse driving scenes.
    \item \textbf{Sample Size Sensitivity:} FID estimates are biased for small sample sizes. Our evaluation uses 200 validation clips, which may not fully capture the distribution.
    \item \textbf{Frame Independence:} FID treats each frame independently and does not capture temporal relationships.
    \item \textbf{Inception Bias:} The Inception-v3 network was trained on ImageNet, which has limited overlap with driving scenes. Features may not capture domain-specific quality aspects.
\end{itemize}

\subsection{FVD Limitations}
\label{subsec:fvd_limitations}

FVD partially addresses the temporal limitation of FID but introduces its own issues:
\begin{itemize}
    \item \textbf{I3D Action Bias:} The I3D network was trained for action recognition on Kinetics-400. Its features are optimized for human actions, not driving scenarios. Vehicle motion, road geometry, and traffic patterns may be underrepresented in the feature space.
    \item \textbf{Fixed Temporal Window:} I3D processes clips of a fixed length, which may not align with the 25-frame clips used in our experiments, requiring padding or subsampling.
    \item \textbf{Computational Cost:} FVD computation requires extracting I3D features from all generated and real videos, which is computationally expensive for large-scale evaluation.
\end{itemize}

\subsection{mIoU for VAE Evaluation}
\label{subsec:miou_limitations}

While mIoU is the standard metric for semantic segmentation, its use for evaluating a VAE's reconstruction quality has a subtle limitation: it equally weights all classes regardless of their visual importance. A class that is visually critical (e.g., cars) but spatially small may have the same mIoU weight as a less important but spatially dominant class (e.g., sky). Weighted mIoU variants could address this, but we use the standard unweighted version for comparability with the literature.

% -----------------------------------------------------------------
\section{Comparison with Related Work}
\label{sec:comparison}

\subsection{Comparison with Ctrl-V}
\label{subsec:comparison_ctrlv}

Our approach extends Ctrl-V~\cite{luo2025ctrlv} by replacing bounding-box conditioning with semantic segmentation maps. Table~\ref{tab:comparison_ctrlv} summarizes the key differences.

\begin{table}[t]
    \centering
    \caption{Comparison between Ctrl-V (original) and our semantic extension}
    \label{tab:comparison_ctrlv}
    \begin{tabular}{lcc}
        \toprule
        \textbf{Aspect} & \textbf{Ctrl-V (BBox)} & \textbf{Ours (Semantic)} \\
        \midrule
        Control signal & 2D/3D bounding boxes & Semantic segmentation maps \\
        Control density & Sparse (box outlines) & Dense (per-pixel labels) \\
        Scene control & Object positions only & Full scene layout \\
        VAE requirement & Standard RGB VAE & Semantic-Native VAE \\
        Stage~1 output & BBox trajectory video & Semantic video \\
        Stage~2 input & BBox frames & Semantic frames \\
        Background control & None & Full (road, vegetation, sky) \\
        Object shape control & None (box only) & Precise silhouettes \\
        \bottomrule
    \end{tabular}
\end{table}

The key advantage of semantic conditioning is \textbf{dense scene control}: while bounding boxes specify only the positions and extents of objects, semantic maps specify the class of every pixel, including background elements (roads, buildings, vegetation, sky) that bounding boxes cannot control. This enables more precise and comprehensive scene specification, which is particularly valuable for synthetic data generation where pixel-level annotations are required.

The trade-off is increased complexity: our approach requires a custom Semantic VAE to bridge the gap between discrete semantic labels and the continuous latent space, whereas Ctrl-V uses the standard RGB VAE to encode bounding box images (which are simply colored rectangles on a black background).

\subsection{Comparison with GAN-Based Methods}
\label{subsec:comparison_gans}

GAN-based methods for driving video generation, such as SVS GAN~\cite{swerdlow2024svsgan}, offer an alternative paradigm. Key differences include:
\begin{itemize}
    \item \textbf{Training Stability:} Diffusion models exhibit more stable training compared to GANs, which suffer from mode collapse and require careful hyperparameter tuning.
    \item \textbf{Sample Diversity:} Diffusion models naturally produce diverse samples through different noise initializations, while GANs may mode-collapse to a limited set of outputs.
    \item \textbf{Inference Speed:} GANs generate samples in a single forward pass, while diffusion models require multiple denoising steps (30 in our case). This makes GANs significantly faster at inference time.
    \item \textbf{Quality:} Recent diffusion models generally achieve better FID and FVD scores than GANs on complex datasets, consistent with the broader trend in generative modeling~\cite{dhariwal2021diffusion}.
\end{itemize}

% -----------------------------------------------------------------
\section{Conclusion}
\label{sec:conclusion}

This chapter has analyzed the experimental results, revealing that:

\begin{enumerate}
    \item The Semantic-Native VAE achieves 89.7\% mIoU through the combination of bypassing the RGB bottleneck, boundary-weighted loss, and Dice loss---each addressing a distinct failure mode of naive approaches.
    
    \item The two-stage diffusion pipeline achieves competitive FID (35.74) and FVD (392.10) scores, demonstrating that semantic maps provide an effective control signal for video generation.
    
    \item Training instabilities related to gradient flow, semantic ID remapping, and memory management were identified and resolved, providing practical insights for future work.
    
    \item Standard evaluation metrics (FID, FVD) have domain-specific limitations when applied to driving video generation, suggesting that task-specific metrics (e.g., downstream perception performance) may better capture generation quality.
\end{enumerate}

% =================================================================
% CHAPTER 6: SUMMARY AND FUTURE WORK
% =================================================================
\chapter{Summary and Future Work}
\label{ch:summary}

% -----------------------------------------------------------------
\section{Summary}
\label{sec:summary}

This thesis presented a two-stage pipeline for semantically controlled video generation of autonomous driving scenes using latent diffusion models. The work was motivated by the need for controllable synthetic data generation, where semantic segmentation maps provide dense, interpretable, and pixel-level control over generated video content.

The key contributions and findings are summarized as follows:

\paragraph{Semantic-Native VAE.}
A novel Variational Autoencoder architecture was designed to bridge the gap between discrete semantic label maps and the continuous latent space of a pretrained Stable Video Diffusion model. By bypassing the 3-channel RGB bottleneck and introducing trainable Semantic Stem (19$\rightarrow$64$\rightarrow$128 channels) and Semantic Head (128$\rightarrow$64$\rightarrow$19 channels) modules around a frozen VAE core, the model achieves 89.7\% mean Intersection-over-Union on KITTI-360 validation data with only $\sim$200K trainable parameters. This represents a 35.4\% absolute improvement over the naive RGB baseline approach.

\paragraph{Loss Function Design.}
The combination of boundary-weighted cross-entropy ($\alpha = 4.0$) and Dice loss ($\lambda = 0.5$) was shown to be critical for semantic reconstruction quality. Boundary weighting alone contributed +14.7\% mIoU improvement in the adapter model, while Dice loss provided an additional +1.7\% improvement in the native architecture by addressing class imbalance---particularly for rare classes such as traffic signs (82.2\% IoU) and poles (84.0\% IoU).

\paragraph{Two-Stage Diffusion Pipeline.}
The complete pipeline consists of:
\begin{itemize}
    \item \textbf{Stage~1:} A modified SVD backbone that predicts future semantic video sequences from an initial RGB frame, trained with the \texttt{--use\_segmentation} flag on KITTI-360 data.
    \item \textbf{Stage~2:} A ControlNet-conditioned video diffusion model that generates photorealistic RGB videos guided by semantic maps, with only the ControlNet weights being trainable.
\end{itemize}
The pipeline achieves a Fr\'echet Inception Distance of 35.74 and a Fr\'echet Video Distance of 392.10 on KITTI-360, demonstrating competitive quality for driving video generation.

\paragraph{Practical Insights.}
The development process revealed several practical insights: (1) gradient flow through frozen layers requires careful management of PyTorch's autograd graph; (2) semantic ID remapping from sparse KITTI-360 labels to continuous training IDs is critical; (3) per-frame decoding with temporal 3D convolution heads provides a practical compromise between memory efficiency and temporal consistency; and (4) parallel training of Stage~1 and Stage~2 (using ground-truth semantic conditioning for Stage~2) significantly reduces total training time.

% -----------------------------------------------------------------
\section{Limitations}
\label{sec:limitations}

Despite the promising results, several limitations should be acknowledged:

\begin{itemize}
    \item \textbf{Resolution:} Training was performed at $192 \times 704$ resolution, which is lower than the original KITTI-360 resolution ($376 \times 1408$) and modern autonomous driving requirements. Higher resolutions would require more GPU memory and longer training times.
    
    \item \textbf{Dataset Scope:} All experiments were conducted on KITTI-360, which captures suburban driving in Karlsruhe, Germany. Generalization to other geographic regions, weather conditions, and driving scenarios remains to be evaluated.
    
    \item \textbf{Temporal Length:} The pipeline generates clips of 25 frames ($\sim$1 second at 25 FPS). Longer video generation would require either autoregressive extension or architectural changes to handle longer sequences.
    
    \item \textbf{Inference Speed:} The two-stage diffusion pipeline with 30 denoising steps per stage is computationally expensive at inference time, making real-time generation infeasible.
    
    \item \textbf{Evaluation Metrics:} FID and FVD, while standard, have known biases (ImageNet/Kinetics-400 feature extractors) that may not fully capture driving-specific quality. Task-specific downstream evaluation (e.g., training a perception model on generated data) would provide a more meaningful assessment.
    
    \item \textbf{Rare Class Performance:} While the Semantic VAE achieves strong performance on most classes, rare and small objects (motorcycle: 55\%, bicycle: 60\%) still have lower IoU compared to dominant classes, suggesting room for improvement in handling extreme class imbalance.
\end{itemize}

% -----------------------------------------------------------------
\section{Future Work}
\label{sec:future_work}

Several directions for future research emerge from this work:

\paragraph{Higher Resolution Training.}
Scaling the pipeline to higher resolutions ($384 \times 1408$ or beyond) using techniques such as patch-based training, multi-resolution architectures, or more memory-efficient attention mechanisms would improve the practical utility of generated videos.

\paragraph{Multi-Dataset Training.}
Training on multiple driving datasets (Cityscapes~\cite{cordts2016cityscapes}, nuScenes, Waymo Open Dataset) would improve generalization and expose the model to diverse driving conditions, camera setups, and geographic environments.

\paragraph{Temporal Extension.}
Extending the pipeline to generate longer video sequences through autoregressive generation (using the last frame of one clip as the first frame of the next) or through latent interpolation techniques would increase the practical applicability.

\paragraph{Downstream Task Evaluation.}
Evaluating the generated videos by training downstream perception models (object detection, semantic segmentation, depth estimation) on synthetic data and measuring performance on real-world test sets would provide the most meaningful assessment of generation quality.

\paragraph{Interactive Control.}
Extending the pipeline to support interactive editing---where a user can modify the semantic map of specific frames and regenerate only the affected portions of the video---would enable creative applications beyond data augmentation.

\paragraph{Conditional Semantic Generation.}
Improving Stage~1 to accept richer conditioning signals (e.g., text descriptions of desired scene changes, trajectory waypoints, or high-level scene graphs) would enable more flexible control over the generated semantic sequences.

\paragraph{Improved VAE Architecture.}
Exploring more sophisticated Semantic VAE designs---such as multi-scale stem and head modules, attention-based boundary refinement, or class-conditional normalization layers---could further improve semantic reconstruction quality, particularly for rare and small classes.

\paragraph{Real-Time Inference.}
Applying diffusion distillation techniques (progressive distillation, consistency models) to reduce the number of denoising steps from 30 to 1--4 would bring the pipeline closer to real-time applicability.


% =================================================================
% APPENDIX
% =================================================================
\appendix
\chapter{Acronyms}
\label{app:acronyms}

\begin{table}[h]
    \centering
    \begin{tabular}{ll}
        \toprule
        \textbf{Acronym} & \textbf{Full Form} \\
        \midrule
        AV & Autonomous Vehicle \\
        BBox & Bounding Box \\
        CE & Cross-Entropy \\
        CFG & Classifier-Free Guidance \\
        CLIP & Contrastive Language-Image Pre-training \\
        CNN & Convolutional Neural Network \\
        DDPM & Denoising Diffusion Probabilistic Model \\
        EDM & Elucidating the Design space of Diffusion Models \\
        ELBO & Evidence Lower Bound \\
        FID & Fr\'echet Inception Distance \\
        FP16 & 16-bit Floating Point (Half Precision) \\
        FVD & Fr\'echet Video Distance \\
        GAN & Generative Adversarial Network \\
        GN & Group Normalization \\
        GPU & Graphics Processing Unit \\
        I2V & Image-to-Video \\
        I3D & Inflated 3D ConvNet \\
        IoU & Intersection-over-Union \\
        KL & Kullback--Leibler \\
        LDM & Latent Diffusion Model \\
        LPIPS & Learned Perceptual Image Patch Similarity \\
        mIoU & Mean Intersection-over-Union \\
        ODE & Ordinary Differential Equation \\
        PSNR & Peak Signal-to-Noise Ratio \\
        SDE & Stochastic Differential Equation \\
        SiLU & Sigmoid Linear Unit \\
        SSIM & Structural Similarity Index \\
        SVD & Stable Video Diffusion \\
        SVD-XT & Stable Video Diffusion -- Extended (25 frames) \\
        VAE & Variational Autoencoder \\
        VRAM & Video Random Access Memory \\
        \bottomrule
    \end{tabular}
\end{table}


% -------------------> end writing here <------------------------
% *****************************************************************
\listoffigures
\listoftables

\ifthenelse{\equal{\doclang}{german}}{
	\bibliographystyle{IEEEtran_ISSger}
}{
	\bibliographystyle{IEEEtran_ISS}
}
\bibliography{refs}

% *****************************************************************
%% Additional page with Declaration ("Eidesstattliche Erklrung");
%% completed automatically
\begin{titlepage}
      \vfill
      \LARGE \ifthenelse{\equal{\doclang}{german}}{\textbf{Erkl\"arung}}{\textbf{Declaration}}
      \vfill

      \ifthenelse{\equal{\doclang}{german}}{
         Hiermit erkl\"are ich, dass ich diese Arbeit selbstst\"andig verfasst und keine anderen als die angegebenen
         Quellen und Hilfsmittel benutzt habe.
      }
      {
         Herewith, I declare that I have developed and written the enclosed thesis entirely by myself and that I have not used sources or means except those declared.
      }

      \vspace{1cm}

      \ifthenelse{\equal{\doclang}{german}}{
         Die Arbeit wurde bisher keiner anderen Pr\"ufungsbeh\"orde vorgelegt und auch noch nicht ver\"offentlicht.
      }
      {
         This thesis has not been submitted to any other authority to achieve an academic grading and has not been published elsewhere.
      }

      \vfill

      
      Stuttgart, \signagedate
      \hfill
      \begin{tabular}{l}
          \hline
          \student
      \end{tabular}
\end{titlepage}



\end{document}
